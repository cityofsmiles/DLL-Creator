 \begin{center}
\textbf{Lesson 4.1.2: Calculating the Mean Deviation for Ungrouped Data}
\end{center}

\vspace*{1ex}

\noindent\textbf{Mean Deviation}: the average of the absolute deviations from the mean

%\[
{\centering $ \text{Mean Deviation} = \dfrac{\sum |x_i - \bar{x}|}{n} $\par}
%\]
where:
\begin{itemize}
    \item $x_i$ are the data values,
    \item $\bar{x}$ is the mean of the data,
    \item $n$ is the number of data values, and
    \item $|x_i - \bar{x}|$ represents the absolute deviation of each data value from the mean.
\end{itemize}

\noindent\textbf{Steps for Calculating the Mean Deviation}

\begin{enumerate}
    \item \textbf{Find the mean:} Add all the data values and divide by the number of values.
    \item \textbf{Find the absolute deviations:} Subtract the mean from each data value and take the absolute value (i.e., ignore any negative signs).
    \item \textbf{Find the mean deviation:} Add all the absolute deviations and divide by the number of data values.
\end{enumerate}



