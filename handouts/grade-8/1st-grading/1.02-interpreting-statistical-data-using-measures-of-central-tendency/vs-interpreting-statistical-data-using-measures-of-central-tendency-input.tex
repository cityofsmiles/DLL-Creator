\begin{center}
\textbf{Lesson 1.2: Interpreting Statistical Data Using Measures of Central Tendency}
\end{center}

\vspace*{1ex}

\begin{itemize}
    \item \textbf{Mean:} can be influenced by outliers (extremely high or low values)
    \item \textbf{Median:} less affected by outliers and provides a better sense of the center in skewed distributions
    \item \textbf{Mode:} useful for categorical data or when a specific value dominates the data set
\end{itemize}

\noindent\textbf{Interpreting the Results}

\noindent\textbf{1. Mean}
When interpreting the mean, consider the following:
\begin{itemize}
%    \item \textbf{Central Tendency:} The mean provides a measure of the central point of the data.
    \item \textbf{Impact of Outliers:} If the mean is significantly higher or lower than most data points, it may indicate the presence of outliers.
 %   \item \textbf{Use Case:} The mean is useful when the data is symmetrically distributed without outliers.
\end{itemize}

\noindent\textbf{2. Median}
When interpreting the median, consider the following:
\begin{itemize}
    \item \textbf{Resistance to Outliers:} The median is a better measure of central tendency when there are outliers.
 %   \item \textbf{Skewed Data:} If the data is skewed (e.g., most values are low, but a few are very high), the median provides a more accurate central point.
 %   \item \textbf{Use Case:} The median is preferred when the data is skewed or when there are outliers.
\end{itemize}

\noindent\textbf{3. Mode}
When interpreting the mode, consider the following:
\begin{itemize}
 %   \item \textbf{Frequency of Occurrence:} The mode tells us which value appears most often, which is useful in understanding the most common category or response.
    \item \textbf{Multiple Modes:} A data set can have more than one mode, indicating multiple common values.
 %   \item \textbf{Use Case:} The mode is useful for categorical data or when identifying the most frequent occurrence.
\end{itemize}
