%\vspace{1ex}
\vspace{0.3ex}
\noindent\textbf{Activity 3.14}

%\vspace{0.75ex}
\vspace{0.2ex}

Use the properties of trapezoids to solve each problem.

\begin{enumerate}[label=\color{blue}\arabic*.]
    \item In trapezoid \(MNOP\), \(MN\) and \(OP\) are parallel with \(MN = 16\) cm and \(OP = 24\) cm. Find the length of the midsegment.
%    \item In isosceles trapezoid \(QRST\), \(QR = 18\) cm, \(ST = 28\) cm, and the height is 9 cm. Find the area of \(QRST\).
    \item In trapezoid \(WXYZ\), \(WX = (4y - 6)\) and \(YZ = 20\) cm. If the midsegment is 15 cm, find \(y\).
    \item In trapezoid \(ABCD\), \(AB\) and \(CD\) are parallel, and the length of the midsegment is 30 cm. If \(AB = 22\) cm, find \(CD\).
%    \item In trapezoid \(EFGH\), \(EF = 10\) cm, \(GH = 35\) cm, and the height is 12 cm. Calculate the area of \(EFGH\).
    \item In trapezoid \(JKLM\), the bases \(JK = (2c + 8)\) cm and \(LM = 40\) cm. If the midsegment is 32 cm, find \(c\).
%    \item In isosceles trapezoid \(UVWX\), \(UV = 20\) cm and \(WX = 38\) cm. The height of the trapezoid is 14 cm. Calculate the area.
    \item In trapezoid \(RSTU\), \(RS\) and \(TU\) are parallel, and the midsegment is 26 cm. If \(RS = 28\) cm, find \(TU\).
%    \item In trapezoid \(MNOP\), if \(MN = (3d + 4)\) cm and \(OP = 34\) cm, and the midsegment is 20 cm, find \(d\).
%    \item In isosceles trapezoid \(QRST\), \(QR = 12\) cm, \(ST = 32\) cm, and the height is 18 cm. Find the area of \(QRST\).
\end{enumerate}
