\begin{center}
\textbf{Lesson 2.08: Illustrating the Circle and Its Parts}
\end{center}

%\vspace*{1ex}
\vspace*{-1.5ex}


\begin{itemize}
    \item A \textbf{circle} is a set of all points in a plane that are equidistant from a fixed point called the \textbf{center}.
    \item Important parts of a circle:
    \begin{itemize}
        \item \textbf{Center} – The fixed point from which all points on the circle are equidistant.
        \item \textbf{Radius} – A line segment from the center to any point on the circle.
        \item \textbf{Diameter} – A line segment that passes through the center and has both endpoints on the circle. It is twice the radius.
        \item \textbf{Chord} – A line segment with both endpoints on the circle.
        \item \textbf{Secant Line} – A line that intersects the circle at two points.
        \item \textbf{Tangent Line} – A line that touches the circle at exactly one point.
        \item \textbf{Arcs} – A portion of the circle’s circumference.
        \begin{itemize}
            \item \textbf{Minor Arc} – An arc that measures less than $180^\circ$.
            \item \textbf{Major Arc} – An arc that measures more than $180^\circ$.
        \end{itemize}
        \item \textbf{Angles} in a circle:
        \begin{itemize}
            \item \textbf{Central Angle} – An angle whose vertex is at the center of the circle.
            \item \textbf{Inscribed Angle} – An angle whose vertex is on the circle and whose sides contain chords of the circle.
        \end{itemize}
    \end{itemize}
\end{itemize}

