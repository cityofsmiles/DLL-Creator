\begin{center}
\textbf{Lesson 1.14.1: Identifying Values of a, h, and k in the Vertex Form of a Quadratic Function}
\end{center}

%\vspace*{1ex}

The vertex form of a quadratic function is given by:
\vspace*{-2ex}
\[ f(x) = a(x - h)^2 + k \]
\vspace*{-4ex}
\begin{itemize}
    \item \textbf{$a$ (Leading Coefficient)}: The value of $a$ determines the direction and magnitude of the parabola's opening. If $a > 0$, the parabola opens upwards; if $a < 0$, it opens downwards. 
    
    \item \textbf{$h$ (Horizontal Shift)}: The value of $h$ represents the horizontal shift or translation of the parabola along the $x$-axis. If $h > 0$, the graph shifts to the right; if $h < 0$, it shifts to the left. 
    
    \item \textbf{$k$ (Vertical Shift)}: The value of $k$ represents the vertical shift or translation of the parabola along the $y$-axis. If $k > 0$, the graph shifts upwards; if $k < 0$, it shifts downwards. 
\end{itemize}

