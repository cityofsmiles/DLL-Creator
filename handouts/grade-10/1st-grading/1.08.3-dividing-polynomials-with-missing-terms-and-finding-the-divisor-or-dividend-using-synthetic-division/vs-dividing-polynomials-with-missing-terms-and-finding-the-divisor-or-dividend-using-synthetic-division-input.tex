\begin{center}
\textbf{Lesson 1.08.3: Dividing Polynomials with Missing Terms and Finding the Divisor or Dividend Using Synthetic Division}
\end{center}

%\vspace*{1ex}
\vspace*{-1.5ex}

\begin{itemize}
    \item When performing \textbf{synthetic division}, it is important to include \textbf{zero coefficients} for any missing terms in the dividend.
    \item Given a \textbf{dividend} and \textbf{quotient}, the missing \textbf{divisor} can be determined using synthetic division.
    \item Similarly, given a \textbf{divisor} and \textbf{quotient}, the missing \textbf{dividend} can be reconstructed.
    % \item The process of synthetic division follows the same steps:
    %       \begin{itemize}
    %           \item Write the coefficients of the polynomial, ensuring that missing terms are represented with zero coefficients.
    %           \item Use the root of the divisor \(x - c\) in synthetic division.
    %           \item Perform the division step-by-step to find the quotient or missing term.
    %       \end{itemize}
    \item Checking the result by multiplying the quotient and divisor should yield the original dividend.
\end{itemize}
