\begin{center}
\textbf{Lesson 4.03: Illustrating Secant and Cosecant}
\end{center}

%\vspace*{1ex}
\vspace*{-1.5ex}

\noindent \textbf{Secant ($\sec$):} the reciprocal of the cosine; also the ratio of the length of the hypotenuse to the length of the adjacent side

{\centering $
    \sec \theta = \dfrac{\text{Hypotenuse}}{\text{Adjacent Side}}
$\par}

\noindent  \textbf{Cosecant ($\csc$):} the reciprocal of the sine; also the ratio of the length of the hypotenuse to the length of the opposite side

{\centering $
    \csc \theta = \dfrac{\text{Hypotenuse}}{\text{Opposite Side}}
$\par}

% \noindent  \textbf{Procedure to Solve:}
%     \begin{enumerate}
%         \item Identify the angle for which secant or cosecant is being calculated.
%         \item Determine the lengths of the relevant sides (hypotenuse, adjacent, or opposite) relative to the angle.
%         \item Apply the appropriate formula for secant or cosecant.
%         \item Simplify the fraction and, if necessary, convert it to a decimal.
%     \end{enumerate}
