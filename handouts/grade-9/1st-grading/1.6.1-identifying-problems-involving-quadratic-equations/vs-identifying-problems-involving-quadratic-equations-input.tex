\begin{center}
\textbf{Lesson 1.6.1: Identifying Problems Involving Quadratic Equations}
\end{center}

\vspace*{1ex}



%\begin{multicols}{2}
\begin{enumerate}[label = \color{blue}\arabic*. ]
%1
\item \textbf{Recognize a Quadratic Situation:} Look for key features such as:
\begin{itemize}
%1
\item Situations involving area, especially problems looking to maximize or minimize the area.
%2
\item Problems involving projectile motion or other scenarios where something is thrown or shot and gravity influences its motion.
%3
\item Situations where variables are squared or products of different variables are considered.
\end{itemize}
%2
\item \textbf{Translate the Problem into an Equation:} 

\begin{itemize}
%1
\item Define your variables: Clearly identify what each variable represents in the context of the problem. 
%2
\item Set up the equation: 
\begin{itemize}
%1
\item Identify the highest degree of the variable involved (look for squares or products).
%2
\item Equate to zero by moving all terms to one side of the equation (standard form).
\end{itemize}
\end{itemize}
%3
\item \textbf{Formulate the Quadratic Equation:}
\begin{itemize}
%1
\item Gather all information and organize it to see how it fits into the quadratic formula.
%2
\item Construct the equation by placing terms in the correct order ($ax^2 + bx + c = 0$).
\end{itemize}
\end{enumerate}
%\end{multicols} 


