\begin{center}
\textbf{Lesson 2.24: Solving Word Problems Involving Tangents, Secants, Angles, Arcs, and Chords of a Circle}
\end{center}

%\vspace*{1ex}
\vspace*{-1.5ex}

\begin{itemize}
    \item A \textbf{tangent} is a line that touches a circle at exactly one point, called the \textbf{point of tangency}.
    \item A \textbf{secant} is a line that intersects a circle at two points.
    \item The \textbf{angle formed by two secants} intersecting outside a circle is given by:

{\centering $
    \theta = \dfrac{1}{2} ( \text{larger intercepted arc} - \text{smaller intercepted arc} )
$\par}
    \item The \textbf{angle formed by a secant and a tangent} outside the circle follows the same formula.
    \item The \textbf{angle formed by two intersecting chords} inside a circle is given by:

{\centering $
    \theta = \dfrac{1}{2} ( \text{sum of intercepted arcs} )
$\par}
    \item The \textbf{Power Theorems}:
    \begin{itemize}
        \item \textbf{Tangent-Secant Theorem}: $ ( \text{tangent segment} )^2 = ( \text{external secant segment} ) \times ( \text{whole secant segment} )$
        \item \textbf{Intersecting Chords Theorem}: $ ( \text{segment 1} ) \times ( \text{segment 2} ) = ( \text{segment 3} ) \times ( \text{segment 4} )$
    \end{itemize}
\end{itemize}
