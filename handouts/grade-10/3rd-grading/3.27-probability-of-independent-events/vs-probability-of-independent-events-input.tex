\begin{center}
\textbf{Lesson 3.27: Probability of Independent Events}
\end{center}

%\vspace*{1ex}
\vspace*{-1.5ex}

\noindent Two events \( A \) and \( B \) are independent if the occurrence of one event does not affect the probability of the other.

\noindent The probability of independent events occurring together is given by:

{\centering $ P(A \cap B) = P(A) \cdot P(B) $\par}

\noindent If there are more than two independent events \( A_1, A_2, \ldots, A_n \), their joint probability is:

{\centering $ P(A_1 \cap A_2 \cap \ldots \cap A_n) = P(A_1) \cdot P(A_2) \cdot \ldots \cdot P(A_n)$\par}


% \noindent\textbf{Step-by-Step Procedure:}

% \begin{enumerate}
%     \item Identify the independent events \( A \) and \( B \) (or more events).
%     \item Compute the probability of each individual event.
%     \item Multiply the probabilities of the events to find \( P(A \cap B) \) or \( P(A_1 \cap A_2 \cap \ldots \cap A_n) \).
% \end{enumerate}
