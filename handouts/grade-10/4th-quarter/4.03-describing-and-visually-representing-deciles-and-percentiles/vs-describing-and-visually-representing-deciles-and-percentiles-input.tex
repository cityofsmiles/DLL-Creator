\begin{center}
\textbf{Lesson 4.03: Describing and Visually Representing Deciles and Percentiles}
\end{center}

%\vspace*{1ex}
\vspace*{-1.5ex}

\noindent \textbf{Deciles} divide a data set into ten equal parts. Each decile represents 10\% of the data.  
    % \begin{itemize}
    %     \item The first decile (\(D_1\)) is the value below which 10\% of the data falls.  
    %     \item The fifth decile (\(D_5\)) is equivalent to the median.  
    %     \item The tenth decile (\(D_{10}\)) is the maximum value of the data set.  
    % \end{itemize}

\noindent \textbf{Percentiles} divide a data set into 100 equal parts.  
    % \begin{itemize}
    %     \item The \(P_{25}\) percentile corresponds to the first quartile (\(Q_1\)).  
    %     \item The \(P_{50}\) percentile corresponds to the median (\(Q_2\)).  
    %     \item The \(P_{75}\) percentile corresponds to the third quartile (\(Q_3\)).  
    % \end{itemize}


\noindent\textbf{Procedure:}  
\begin{enumerate}
    \item Arrange the data in ascending order.  
    \item Determine the position of the desired decile or percentile using the formula:  

      {\centering $ \text{Position} = \dfrac{k}{n} \cdot (N + 1), $\par}
      
    where \(k\) is the decile or percentile number, \(n\) is the total number of data divisions (10 for deciles, 100 for percentiles), and \(N\) is the number of data points.  
    \item Locate the corresponding value in the data set.  
  %  \item For visual representation, use a line graph or cumulative frequency graph to indicate the position of deciles or percentiles.  
\end{enumerate} 
