\begin{center}
\textbf{Lesson 1.3.1: Nature of Roots of a Quadratic Equation}

\end{center}

\vspace*{-0.5ex}

\noindent\textbf{Types of Numbers}

\textbf{Integers:} consist of natural numbers, zero and the negative of natural numbers 

\textbf{Rational:} numbers which can be expressed as the ratio of two integers

\textbf{Irrational:} numbers which cannot be expressed as the ratio of two integers. They are non-repeating, non-terminating decimals. 

\textbf{Imaginary:} numbers that result in a negative number when squared

\vspace{-1.3ex}

\begin{center}
\textbf{Discriminant Formula}
\end{center} 

\vspace{-1ex}

To determine the nature of the roots of a quadratic equation, we use the discriminant formula: 
\vspace{-0.7ex}
$$ D = b^2 - 4ac $$

\vspace*{-5ex}

\def\colOneWidth{15.5em}
\def\colTwoWidth{18.5em}

{\fontsize{7.5}{8}\selectfont{
\begin{center}
\noindent\begin{tabular}{cc}
%5\hline 
 \thead{\colOneWidth}{Discriminant Value} & \thead{\colTwoWidth}{Nature of Roots}\\ 
%\hline  
  \tCellC{\colOneWidth}{$D > 0$ and a perfect square} & \tCellC{\colTwoWidth}{Two real, rational and unequal roots}  \\ 
  \tCellC{\colOneWidth}{$D > 0$ and not a perfect square} & \tCellC{\colTwoWidth}{Two real, irrational and unequal roots}  \\ 
  \tCellC{\colOneWidth}{$D = 0$} & \tCellC{\colTwoWidth}{Two real, rational and  equal roots}  \\ 
  \tCellC{\colOneWidth}{$D < 0$} & \tCellC{\colTwoWidth}{Two imaginary and unequal roots}  \\ 
\end{tabular}
\end{center}
}}

