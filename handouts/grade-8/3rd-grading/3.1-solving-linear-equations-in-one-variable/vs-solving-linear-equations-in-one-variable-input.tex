 \begin{center}
\textbf{Lesson 3.1: Solving Linear Equations in One Variable}
\end{center}

\vspace*{1ex}

\noindent\textbf{Linear equation:} an equation where the variable is raised to the power of one, and the equation represents a straight line when graphed. 

\noindent\textbf{Steps to Solving Linear Equations in One Variable}
%\begin{multicols}{2}
\setlist{nolistsep}
\begin{enumerate}[noitemsep, label = \color{blue}\arabic*. ]
    \item \textbf{Simplify both sides of the equation:} Combine like terms and remove any parentheses by using the distributive property.
    \item \textbf{Move the variable terms to one side:} Use addition or subtraction to get all terms with the variable on one side of the equation.
    \item \textbf{Move the constant terms to the other side:} Use addition or subtraction to get all constant terms on the other side.
    \item \textbf{Solve for the variable:} Isolate the variable by dividing or multiplying both sides of the equation by the coefficient of the variable.
    \item \textbf{Check your solution:} Substitute your solution back into the original equation to ensure it makes the equation true.
\end{enumerate}



