%\vspace{1ex}
\vspace{0.3ex}
\noindent\textbf{Practice Exercises 2.27}

%\vspace{0.75ex}
\vspace{0.2ex}

Use the distance and/or midpoint formula to prove the given geometric properties.

\begin{enumerate}
   % \item Show that the triangle with vertices $A(1,2)$, $B(5,6)$, and $C(9,2)$ is isosceles.
    \item Prove that the quadrilateral with vertices $P(0,1)$, $Q(4,5)$, $R(8,1)$, and $S(4,-3)$ is a rhombus.
 %   \item Find the midpoint of the diagonal of the rectangle with vertices at $(-2,1)$ and $(6,9)$.
    \item Show that the line segment joining $A(-5,-2)$ and $B(3,4)$ is bisected by point $M(-1,1)$.
    %\item Prove that the triangle with vertices $X(2,3)$, $Y(6,7)$, and $Z(2,11)$ is isosceles.
    \item Determine whether the quadrilateral with vertices $K(-3,-2)$, $L(5,6)$, $M(11,-2)$, and $N(5,-8)$ is a parallelogram.
    %\item Prove that the line segment with endpoints $C(0,-1)$ and $D(6,7)$ is bisected by point $E(3,3)$.
    \item Show that the rectangle with opposite corners at $(-6,4)$ and $(8,10)$ has a diagonal midpoint at $(1,7)$.
    %\item Verify that the triangle with vertices $P(1,0)$, $Q(5,4)$, and $R(9,0)$ is isosceles.
    \item Prove that the quadrilateral with vertices $A(2,1)$, $B(8,5)$, $C(14,1)$, and $D(8,-3)$ is a rhombus.
\end{enumerate}
