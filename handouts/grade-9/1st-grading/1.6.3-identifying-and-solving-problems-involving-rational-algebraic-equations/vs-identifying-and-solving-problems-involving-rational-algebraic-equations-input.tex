\begin{center}
\textbf{Lesson 1.6.3: Identifying and Solving Problems Involving Rational Algebraic Equations}
\end{center}

\vspace*{1ex}

%\begin{multicols}{2}
\begin{enumerate}[label = \color{blue}\arabic*. ]
%1
\item \textbf{Read the Problem Carefully:} Understand the problem and identify the key information provided.
%2
\item \textbf{Identify Variables:} Determine what the unknown variable(s) represent in the problem. Assign variables to represent the quantities involved.
%3
\item \textbf{Set Up the Equation:} Translate the problem into an equation. Identify the rational expressions involved and write them as fractions. Set the expression equal to zero if necessary.
%4
\item \textbf{Find Common Denominators:} If there are multiple fractions involved, find the common denominator and rewrite the equation to eliminate fractions.
\item \textbf{Transform into a Quadratic Equation:} If the equation is not already quadratic, rearrange it to form a quadratic equation. This may involve multiplying both sides by the common denominator to clear fractions.
\item \textbf{Solve the Quadratic Equation:} Use appropriate methods to solve the quadratic equation. This may include factoring, completing the square, or using the quadratic formula.
\item \textbf{Check Solutions:} Verify the solutions by substituting them back into the original equation. Ensure that the solutions satisfy any restrictions on the variables (e.g., avoiding division by zero).
\item \textbf{Interpret the Solutions:} Interpret the solutions in the context of the original problem. Make sure the solutions make sense given the problem's constraints.
\end{enumerate}
%\end{multicols} 



