\begin{center}
\textbf{Lesson 2.04: Graphing Polynomial Functions Using Intercepts and Zeros}
\end{center}

%\vspace*{1ex}
\vspace*{-1.5ex}

\begin{itemize}
    \item To sketch the graph of a \textbf{polynomial function}, we analyze its \textbf{intercepts}, \textbf{key points}, and \textbf{end behavior}.
    \item The \textbf{x-intercepts} (or \textbf{zeros}) are the values of \( x \) where the function equals zero, found by solving \( P(x) = 0 \).
    \item The \textbf{y-intercept} is the point where the graph crosses the \( y \)-axis, found by evaluating \( P(0) \).
    \item The \textbf{multiplicity} of a zero affects how the graph behaves at that zero:
    \begin{itemize}
        \item If a zero has \textbf{odd} multiplicity, the graph \textbf{crosses} the x-axis at that point.
        \item If a zero has \textbf{even} multiplicity, the graph \textbf{touches} the x-axis and turns around.
    \end{itemize}
    \item The \textbf{Table of Signs} is used to determine where the function is positive or negative by testing points between the zeros.
    \item Using the \textbf{Leading Coefficient Test}, we determine the \textbf{end behavior} of the polynomial.
    \item By plotting key points and considering the above properties, we can \textbf{sketch an accurate graph}.
\end{itemize}
