\begin{center}
\textbf{Lesson 1.13.1: Graphing Quadratic Functions}
\end{center}

%\vspace*{1ex}

%\textbf{Key Features of Quadratic Functions}

\begin{itemize}
    \item \textbf{Vertex:} the highest or lowest point on the graph of a quadratic function
    \item \textbf{Axis of Symmetry:} a vertical line passing through the vertex, dividing the parabola into two symmetrical halves.
    \item \textbf{Direction of Opening:} The direction of opening of the parabola depends on the sign of the coefficient \(a\). If \(a > 0\), the parabola opens upwards, and if \(a < 0\), the parabola opens downwards.
    \item \textbf{Intercepts:} The x-intercepts (zeros) are the points where the graph intersects the x-axis, and the y-intercept is the point where the graph intersects the y-axis.
\end{itemize}

\textbf{Steps for Graphing Quadratic Functions}

\begin{enumerate}
    \item \textbf{Find the Vertex:} Find the vertex $(h, k)$ and the line of symmetry $x=h$ by expressing the function in the form of $y = a(x - h)^{2} + k$.
    \item \textbf{Find the Intercepts:} To find the x-intercepts, set \( f(x) = 0 \) and solve for \( x \). To find the y-intercept, evaluate \( f(0) \).
    \item \textbf{Plot Points and Draw the Parabola:} Plot the vertex, intercepts, and additional points if needed. Then, draw a smooth curve through the points to represent the parabola.
\end{enumerate}

