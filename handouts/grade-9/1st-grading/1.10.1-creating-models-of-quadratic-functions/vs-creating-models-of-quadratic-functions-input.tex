\begin{center}
\textbf{Lesson 1.10.1: Creating Models of Quadratic Functions}
\end{center}

\vspace*{1ex}
\textbf{Quadratic function:} any function of the form $ y = ax^{2} + bx + c $ or $ f(x) = ax^{2} + bx + c $
where $a$, $b$ and $c$ are real numbers and $a$ is not 
equal to 0. 

\textbf{How to Create Models of Quadratic Functions}
\begin{enumerate}
    \item \textbf{Identify the Variables:}
    \begin{itemize}
        \item Determine the quantities involved in the problem and assign variables to represent them.
        \item Clearly define what each variable represents in the context of the problem.
    \end{itemize}
    
    \item \textbf{Write the Quadratic Function:}
    \begin{itemize}
        \item Choose a quadratic function that represents the relationship between the variables.
        \item The general form of a quadratic function is \( f(x) = ax^2 + bx + c \), where \( a \), \( b \), and \( c \) are constants.
    \end{itemize}
    
    \item \textbf{Determine the Parameters:}
    \begin{itemize}
        \item Use the given information or data to determine the values of the coefficients \( a \), \( b \), and \( c \) in the quadratic function.
    \end{itemize}
    
    \item \textbf{Test the Model:}
    \begin{itemize}
        \item Use the quadratic function to make predictions or calculations based on the model.
        \item Compare the model's predictions to real-world observations or data to assess its accuracy.
    \end{itemize}
\end{enumerate}

