\begin{center}
\textbf{Lesson 2.21: Segments of Secants Power Theorem and Finding Unknown Lengths}
\end{center}

%\vspace*{1ex}
\vspace*{-1.5ex}

\begin{itemize}
    \item A \textbf{secant} is a line that intersects a circle at two points.
    \item When two secants are drawn from an external point to a circle, their segments satisfy the \textbf{Segments of Secants Power Theorem}.
    \item The theorem states that if two secants, $\overline{AP}$ and $\overline{BP}$, intersect the circle at points $C$ and $D$ respectively, then:

{\centering $
    AP \cdot AC = BP \cdot BD
$\par}
    \item This theorem allows us to find unknown segment lengths when two secants intersect outside a circle.
\end{itemize}
