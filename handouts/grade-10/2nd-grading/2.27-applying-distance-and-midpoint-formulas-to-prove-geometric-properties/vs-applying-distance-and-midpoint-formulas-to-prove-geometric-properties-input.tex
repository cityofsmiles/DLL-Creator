\begin{center}
\textbf{Lesson 2.27: Applying Distance and Midpoint Formulas to Prove Geometric Properties}
\end{center}

%\vspace*{1ex}
\vspace*{-1.5ex}

\begin{itemize}
    \item The \textbf{distance formula} is used to find the length of a segment between two points $(x_1, y_1)$ and $(x_2, y_2)$:

{\centering $
    d = \sqrt{(x_2 - x_1)^2 + (y_2 - y_1)^2}
$\par}
    \item The \textbf{midpoint formula} is used to find the midpoint $M$ of a segment with endpoints $(x_1, y_1)$ and $(x_2, y_2)$:

{\centering $
    M \left( \dfrac{x_1 + x_2}{2}, \dfrac{y_1 + y_2}{2} \right)
$\par}
    \item These formulas help prove geometric properties such as:
    \begin{itemize}
        \item Whether a triangle is \textbf{isosceles} (by showing two sides are equal).
        \item Whether a quadrilateral is a \textbf{parallelogram, rectangle, rhombus}, or \textbf{square} (by proving side lengths, diagonals, or midpoints are equal).
        \item Whether a point is on the \textbf{perpendicular bisector} of a segment.
    \end{itemize}
\end{itemize}
