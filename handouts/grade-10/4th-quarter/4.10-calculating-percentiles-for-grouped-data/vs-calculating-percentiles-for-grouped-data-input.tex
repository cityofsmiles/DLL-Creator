\begin{center}
\textbf{Lesson 4.10: Calculating Percentiles for Grouped Data}
\end{center}

%\vspace*{1ex}
\vspace*{-1.5ex}

\noindent \textbf{Percentiles} divide a data set into \(100\) equal parts, each containing \(1\%\) of the data. For grouped data, percentiles are calculated using the formula:  
\[
P_k = L + \dfrac{\left(\dfrac{kN}{100} - CF\right)}{f} \cdot h
\]
where:  
\begin{itemize}
    \item \(P_k\) = \(k\)-th percentile (\(k = 1, 2, \dots, 99\))
    \item \(L\) = Lower boundary of the percentile class
    \item \(N\) = Total frequency
    \item \(CF\) = Cumulative frequency before the percentile class
    \item \(f\) = Frequency of the percentile class
    \item \(h\) = Class width
\end{itemize}

% \noindent\textbf{Procedure:}  
% \begin{enumerate}
%     \item Determine the percentile class by locating the cumulative frequency interval that contains \(\dfrac{kN}{100}\).  
%     \item Identify the lower boundary (\(L\)), cumulative frequency before the percentile class (\(CF\)), frequency of the percentile class (\(f\)), and class width (\(h\)).  
%     \item Substitute the values into the formula to calculate \(P_k\).  
% \end{enumerate}
