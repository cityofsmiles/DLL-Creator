\begin{center}
\textbf{Lesson 2.13: Theorems on Central Angles, Arcs, and Chords}
\end{center}

%\vspace*{1ex}
\vspace*{-1.5ex}

\begin{itemize}
    \item A \textbf{central angle} is an angle whose vertex is at the center of a circle, and its sides are radii.
    \item The measure of a \textbf{central angle} is equal to the measure of its intercepted \textbf{arc}.
    \item A \textbf{chord} is a segment whose endpoints lie on the circle.
    \item If two chords in a circle are congruent, then they determine congruent arcs.
    \item A perpendicular from the center of a circle to a chord bisects the chord.
    \item If a radius of a circle bisects a chord that is not a diameter, then it is perpendicular to the chord.
    \item Chords equidistant from the center of a circle are congruent.
    \item Congruent chords in a circle are equidistant from the center.
\end{itemize}
