\begin{center}
\textbf{Lesson 1.1.2: Median and Mode for Ungrouped Data}
\end{center}

%\vspace*{1ex}

\textbf{Median}: the middle value in a set of data arranged according to size/magnitude (either increasing or decreasing) 

To find the median of an ungrouped data, use the formula
\vspace*{-1ex}
\[
\tilde{x} = \displaystyle \frac{1}{2}(N+1)th
\] 

\vspace*{-1ex}
\textbf{Mode}
\begin{itemize} 
\item the measure or value which occurs most frequently in a set of data
\item the value with the greatest frequency
\end{itemize}  

To find the mode for a set of data:
\begin{enumerate}[label = \arabic*. ]
%1
\item select the measure that appears most often in the set; 
%2
\item if two or more measures appear the same number of times, then each of these 
values is a mode; and
%3
\item if every measure appears the same number of times, then the set of data has no 
mode.
\end{enumerate}   

