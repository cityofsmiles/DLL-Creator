\begin{center}
\textbf{Lesson 1.8.1: Solving Quadratic Inequalities}
\end{center}

\vspace*{1ex}

%\begin{multicols}{2}
\begin{enumerate}[label = \color{blue}\arabic*. ]
%1
\item Express the quadratic inequality as a quadratic \\equation in the form of 
$ ax^{2} + bx + c = 0 $ and then solve for $x$. 
%2
\item Locate the numbers found in step one on a number line. They serve as the boundary points. The number line will be divided into regions.
%3
\item Choose one number from each region as a test point. Substitute the test point to the original inequality. 
%4
\item If the inequality holds true for the test point, then that region belongs to the solution set, otherwise, it is not part of the solution set of the inequality.
%5
\item Write the solution set as interval notation.
\end{enumerate}
%\end{multicols} 





