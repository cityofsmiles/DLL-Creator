\begin{center}
\textbf{Lesson 4.07: Illustrating Trigonometric Ratios of Reflex Angles}
\end{center}

%\vspace*{1ex}
\vspace*{-1.5ex}

\noindent \textbf{Reflex Angles:} angles greater than \(180^\circ\) but less than \(360^\circ\)

\noindent  The trigonometric ratios of reflex angles are based on their reference angles:

\vspace*{-1em}

\begin{minipage}[c]{0.48\textwidth}
\begin{align*}
        \sin(180^\circ + \theta) &= -\sin \theta \\
        \cos(180^\circ + \theta) &= -\cos \theta \\
        \tan(180^\circ + \theta) &= \tan \theta 
\end{align*}
\end{minipage}
\begin{minipage}[c]{0.48\textwidth}
\begin{align*}
         \csc(180^\circ + \theta) &= -\csc \theta \\
        \sec(180^\circ + \theta) &= -\sec \theta \\
        \cot(180^\circ + \theta) &= \cot \theta
    \end{align*}
\end{minipage}

 \noindent     \textbf{Procedure:}
    \begin{enumerate}
        \item Identify the given reflex angle and calculate the reference angle (\( \text{ref angle} = \text{reflex angle} - 180^\circ \)).
        \item Apply the appropriate trigonometric formula for the given reflex angle.
        \item Simplify the expression.
    \end{enumerate}
