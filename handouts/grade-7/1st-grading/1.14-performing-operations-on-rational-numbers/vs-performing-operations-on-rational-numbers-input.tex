\begin{center}
\textbf{Lesson 1.14: Performing Operations on Rational Numbers}
\end{center}

%\vspace*{1ex}
\vspace*{-1.5ex}

\begin{itemize}
  %  \item A \textbf{rational number} is any number that can be expressed as \(\dfrac{a}{b}\), where \(a\) and \(b\) are integers and \(b \neq 0\).
    \item \textbf{Addition and Subtraction:}
    \begin{itemize}
        \item Convert all numbers to a common denominator before adding or subtracting.
        \item If dealing with decimals, align decimal points before performing the operation.
        \item Remember the rules of signs: 
        \begin{itemize}
            \item Adding two numbers with the same sign keeps the sign.
            \item Adding two numbers with different signs subtracts their absolute values and takes the sign of the larger number.
        \end{itemize}
    \end{itemize}
    \item \textbf{Multiplication:}
    \begin{itemize}
        \item Multiply the numerators together and the denominators together.
        \item Simplify the fraction, if necessary.
        \item If multiplying decimals, count the total number of decimal places in the factors to place the decimal in the product.
    \end{itemize}
    \item \textbf{Division:}
    \begin{itemize}
        \item Multiply by the reciprocal of the divisor.
        \item Simplify the resulting fraction.
    \end{itemize}
\end{itemize}
