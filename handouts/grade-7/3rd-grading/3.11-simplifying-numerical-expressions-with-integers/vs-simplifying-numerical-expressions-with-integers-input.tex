\begin{center}
\textbf{Lesson 3.11: Simplifying Numerical Expressions with Integers}
\end{center}

%\vspace*{1ex}
\vspace*{-1.5ex}

\noindent\textbf{Properties of Integers}
%\begin{multicols}{2}
\setlist{nolistsep}
\begin{enumerate}[noitemsep, label = \color{blue}\arabic*. ]
\item Closure Property of Addition and Multiplication: If \(a\) and \(b\) are integers, then \(a + b\) and \(ab\) are also integers.
\item Commutative Property of Addition and Multiplication: If \(a\) and \(b\) are integers, then \(a + b = b + a\), and \(ab = ba\).
\item Associative Property of Addition and Multiplication:  If \(a\), \(b\), and \(c\) are integers, then \(a + (b + c) = (a + b) + c\), and \(a(bc) = (ab)c\).
\item Distributive Property of Multiplication over Addition:  If \(a\), \(b\), and \(c\) are integers, then \(a(b + c) = ab + ac\) or \(a(b - c) = ab - ac\).
\item Division Property of Zero: Whenever zero is divided by any integers (except zer), the quotient is always zero.
\item Identity Property of Addition: The sum of any number and zero is the number itself.
\item Identity Property of Multiplication: The product of any integer and one is the number itself.
\item Inverse Property of Addition: For every integer \(a\), there exists an additive inverse \(-a\) such that the sum of \(a\) and its additive inverse \(-a\) is the additive identity 0; that is, \(a + (-a) = 0\).
\item Inverse Property of Multiplication: For every nonzero integer \(a\), there exists a multiplicative inverse \(\dfrac{1}{a}\) such that the product of \(a\) and its multiplicative inverse \(\dfrac{1}{a}\) is the multiplicative identity 1; that is, \(a \times \dfrac{1}{a} = 1\).
\end{enumerate}
% \end{multicols}

\noindent\textbf{GEMDAS}

If there is a series of operations, the operations should be performed in the following order:
\begin{multicols}{2}
\setlist{nolistsep}
\begin{enumerate}[noitemsep, label = \color{blue}\arabic*. ]
\item Grouping symbols %(parentheses, brackets, braces)
\item Exponents
\item Multiplication or Division
\item Addition or Subtraction
\end{enumerate}
\end{multicols}
