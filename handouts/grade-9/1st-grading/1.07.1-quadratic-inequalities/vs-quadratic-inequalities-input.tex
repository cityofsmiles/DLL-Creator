\vspace*{-5ex}
\begin{center}
\textbf{Lesson 1.7.1: Quadratic Inequalities}
\end{center}

\vspace*{-1ex}

\textbf{Quadratic inequalities:} inequalities that involve quadratic expressions. They are typically in any of the following forms: 
\begin{center}
\setlength{\columnsep}{-3em}
\begin{multicols}{2}				
\( ax^2 + bx + c < 0 \) \\
\( ax^2 + bx + c > 0 \) \\
\( ax^2 + bx + c \leq 0 \) \\
\( ax^2 + bx + c \geq 0 \)
\end{multicols} 
\end{center} 

where \( a \), \( b \), and \( c \) are real numbers and \( a \neq 0 \).

\textbf{Steps for Graphing Quadratic Inequalities}
\begin{enumerate}
    \item \textbf{Write the inequality in standard form:} Ensure the \\quadratic inequality is in the form \( ax^2 + bx + c < 0 \), \( > 0 \), \( \leq 0 \), or \( \geq 0 \).
    \item \textbf{Graph the corresponding quadratic equation:} Plot the parabola for the equation \( y = ax^2 + bx + c \). Identify whether the parabola opens upwards (\( a > 0 \)) or downwards (\( a < 0 \)).
    \item \textbf{Determine the vertex and axis of symmetry:} Use the formulas \( x = -\frac{b}{2a} \) for the axis of symmetry and calculate the vertex point using this \( x \)-value in the equation.
    \item \textbf{Find the x-intercepts (if any):} Solve \( ax^2 + bx + c = 0 \) to find the points where the parabola crosses the x-axis.
    \item \textbf{Test points:} Select test points in the intervals divided by the vertex and x-intercepts to determine where the inequality holds true.
    \item \textbf{Shade the appropriate region:} Based on whether the inequality is \( < 0 \), \( > 0 \), \( \leq 0 \), or \( \geq 0 \), shade inside or outside of the parabola.
\end{enumerate}

