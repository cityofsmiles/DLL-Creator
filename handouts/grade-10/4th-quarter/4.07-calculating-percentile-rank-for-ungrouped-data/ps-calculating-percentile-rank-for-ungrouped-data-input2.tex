%\vspace{1ex}
\vspace{0.3ex}
\noindent\textbf{Activity 4.07}

%\vspace{0.75ex}
\vspace{0.2ex}

For each data set below, calculate the percentile rank of the given value.  

\begin{enumerate}
    \item Percentile rank of \(18\) in the data set: \(4, 8, 12, 16, 20, 24, 28\).  
    \item Percentile rank of \(64\) in the data set: \(10, 20, 30, 40, 50, 60, 70, 80\).  
    \item Percentile rank of \(38\) in the data set: \(9, 18, 27, 36, 45, 54, 63\).  
    \item Percentile rank of \(29\) in the data set: \(3, 6, 12, 18, 27, 33, 39\).  
    \item Percentile rank of \(97\) in the data set: \(30, 60, 90, 120, 150, 180\).  
    \item Percentile rank of \(23\) in the data set: \(5, 10, 15, 20, 25, 30, 35\).  
    \item Percentile rank of \(78\) in the data set: \(15, 30, 45, 60, 75, 90, 105\).  
    \item Percentile rank of \(46\) in the data set: \(11, 22, 33, 44, 55, 66, 77\).  
    \item Percentile rank of \(47\) in the data set: \(8, 16, 24, 32, 40, 48, 56\).  
    \item Percentile rank of \(20\) in the data set: \(3, 6, 9, 12, 15, 19, 21, 27\).  
\end{enumerate}
