\begin{center}
\textbf{Lesson 1.06.3: Finding the Sum of Terms in a Finite Geometric Sequence}
\end{center}

%\vspace*{1ex}
\vspace*{-1.5ex}

\begin{itemize}
    \item A \textbf{geometric sequence} is a sequence where each term is found by multiplying the previous term by a constant ratio \( r \).
    \item The \textbf{sum} of the first \( n \) terms of a geometric sequence (denoted as \( S_n \)) is given by the formula:

{\centering $  S_n = \dfrac{a_1(1 - r^n)}{1 - r}, \quad \text{for } r \neq 1  $\par}
      %\[ S_n = \dfrac{a_1(1 - r^n)}{1 - r}, \quad \text{for } r \neq 1 \]
          where:
          \begin{multicols}{2}
            \begin{itemize}
  
              \item \( S_n \) is the sum of the first \( n \) terms,
              \item \( a_1 \) is the first term,
              \item \( r \) is the common ratio, and
              \item \( n \) is the number of terms.
            
              \end{itemize}
                  \end{multicols}
    \item This formula allows us to find the sum without adding each term individually.
\end{itemize}
