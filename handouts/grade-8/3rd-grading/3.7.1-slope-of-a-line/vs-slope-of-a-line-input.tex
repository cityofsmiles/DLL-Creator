  \begin{center}
\textbf{Lesson 3.7.1: Slope of a Line}
\end{center}

\vspace*{-2ex}

\noindent\textbf{Slope:} the steepness of a line

\noindent\textbf{How to Find the Slope:}
\begin{itemize}
%1
   \item Case 1: If two points on the line are given.\\
   The slope $ m $  of the line passing through two points $ P_1(x_1, y_1) $   and $ P_2(x_2, y_2) $  is given by $ m = \dfrac{y_2-y_1}{x_2-x_1}, \text{ where } x_1 \neq x_2 $.
    	 
%2
   \item Case 2: If the equation is given.\\
   If the linear equation is written in the form $ y = mx + b $, $ m $ is the slope, that is, the slope is always the numerical coefficient of $ x$.
%3
   \item Case 3: If the graph is given.   $ \text{slope} = m = \dfrac{\text{rise}}{\text{run}} = \dfrac{\text{vertical change}}{\text{horizontal change}} $  
\end{itemize}


\noindent\textbf{Trends of the Slope:}
\begin{itemize}
%1
   \item Case 1: If the slope is positive, the trend is increasing.
    	 
%2
   \item Case 2: If the slope is negative, the trend is decreasing.
%3
   \item Case 3: If the slope is zero, the trend is horizontal.
   \item Case 4: If the slope is undefined, the trend is vertical.

\end{itemize}
