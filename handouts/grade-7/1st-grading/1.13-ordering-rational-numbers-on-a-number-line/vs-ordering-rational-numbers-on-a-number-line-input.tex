\begin{center}
\textbf{Lesson 1.13: Ordering Rational Numbers on a Number Line}
\end{center}

%\vspace*{1ex}
\vspace*{-1.5ex}

\begin{itemize}
  %  \item A \textbf{rational number} is any number that can be written as \(\dfrac{a}{b}\), where \(a\) and \(b\) are integers and \(b \neq 0\).
    \item Rational numbers include integers, fractions, and terminating or repeating decimals.
    \item To order rational numbers:
    \begin{itemize}
        \item Convert all numbers to the same form (fractions, decimals, or percentages).
        \item Compare their values.
        \item Arrange them from least to greatest (ascending) or greatest to least (descending).
    \end{itemize}
    \item A \textbf{number line} helps visualize the order of rational numbers:
    \begin{itemize}
        \item Numbers increase from left to right.
        \item Negative numbers are to the left of zero, and positive numbers are to the right.
        \item The closer a number is to zero, the smaller its absolute value.
    \end{itemize}
\end{itemize}
