\begin{center}
\textbf{Lesson 4.06: Calculating Deciles and Percentiles for Ungrouped Data}
\end{center}

%\vspace*{1ex}
\vspace*{-1.5ex}

\noindent \textbf{Deciles:} Deciles divide a data set into 10 equal parts.  
        \begin{itemize}
            \item \(D_k\) (k-th Decile): The value below which \(\dfrac{k}{10}\) of the data lies, where \(k = 1, 2, \ldots, 9\).
        \end{itemize}

\noindent \textbf{Percentiles:} Percentiles divide a data set into 100 equal parts.  
        \begin{itemize}
            \item \(P_k\) (k-th Percentile): The value below which \(\dfrac{k}{100}\) of the data lies, where \(k = 1, 2, \ldots, 99\).
        \end{itemize}

  %       \noindent \textbf{Percentile Rank:} The percentile rank of a value indicates the percentage of data below that value. The percentile rank formula is:

% {\centering $        R = \dfrac{P}{100}(N + 1) $\par}

% \noindent where:
% \begin{itemize}
% \item \(R =\) the rank order of the score
% \item \(P =\) the percentile rank
% \item \(N =\) the number of scores in the distribution
% \end{itemize}
  


% % \noindent\textbf{Procedure:}  
% % \begin{enumerate}
% %     \item Arrange the data in ascending order.  
% %     \item Calculate the position of the desired decile or percentile using the formula:  
% %         \[
% %         L = \dfrac{k(n+1)}{m}
% %         \]
% %         where:
% %         \begin{itemize}
% %             \item \(k\) is the desired decile or percentile,  
% %             \item \(n\) is the number of data points, and  
% %             \item \(m\) is 10 for deciles or 100 for percentiles.  
% %         \end{itemize}
% %     \item Identify the value at the calculated position \(L\). If \(L\) is a whole number, the value at that position is the result. If \(L\) is not a whole number, interpolate between the two nearest data points.  
% % \end{enumerate}
