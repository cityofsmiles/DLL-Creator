 \begin{center}
\textbf{Lesson 2.7: Volume of Cones and Spheres}
\end{center}

\vspace*{-1ex}

\noindent\textbf{Cone:} a three-dimensional geometric shape with a circular base and a single vertex
\setlist{nolistsep}
\begin{itemize}[noitemsep]
    \item \textbf{Base:} The flat circular surface of the cone.
    \item \textbf{Radius (r):} The distance from the center of the base to any point on the circumference.
    \item \textbf{Height (h):} The perpendicular distance from the base to the vertex.
    \item \textbf{Slant Height (l):} The distance from the vertex to any point on the circumference of the base.
    \item \textbf{Vertex:} The point where all the straight lines drawn from the base meet.
\end{itemize}

\noindent\textbf{Sphere:} a perfectly round three-dimensional shape
\begin{itemize}
    \item \textbf{Radius (r):} The distance from the center of the sphere to any point on its surface.
    \item \textbf{Diameter (d):} The distance across the sphere, passing through the center. It is twice the radius (\(d = 2r\)).
    \item \textbf{Surface Area:} The total area that the surface of the sphere occupies.
    \item \textbf{Volume:} The amount of space inside the sphere.
\end{itemize}

%\noindent\textbf{Volume:} the measure of the amount of space occupied by a three-dimensional object. It is measured in cubic units (e.g., cubic centimeters, cubic meters).




