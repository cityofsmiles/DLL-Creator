\begin{center}
\textbf{Lesson 3.28: Probability of Dependent Events}
\end{center}

%\vspace*{1ex}
\vspace*{-1.5ex}

\noindent Two events \( A \) and \( B \) are dependent if the occurrence of one event affects the probability of the other.

\noindent The probability of dependent events occurring together is given by:

{\centering $ P(A \cap B) = P(A) \cdot P(B|A) $\par}

\noindent \( P(B|A) \) is the conditional probability that event \( B \) occurs given that event \( A \) has already occurred.

% \noindent\textbf{Step-by-Step Procedure:}

% \begin{enumerate}
%     \item Identify the dependent events \( A \) and \( B \).
%     \item Compute \( P(A) \), the probability of the first event.
%     \item Determine \( P(B|A) \), the conditional probability of \( B \) given \( A \).
%     \item Multiply \( P(A) \) and \( P(B|A) \) to find \( P(A \cap B) \).
% \end{enumerate}
