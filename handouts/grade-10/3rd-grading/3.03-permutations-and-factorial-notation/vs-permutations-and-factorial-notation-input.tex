\begin{center}
\textbf{Lesson 3.03: Permutations and Factorial Notation}
\end{center}

%\vspace*{1ex}
\vspace*{-1.5ex}


\noindent \textbf{Permutation:} An arrangement of objects where the order is important.

\noindent \textbf{Factorial Notation:} The product of all positive integers from \(1\) to \(n\), denoted by \(n!\)

{\centering $ n! = n \cdot (n-1) \cdot (n-2) \cdots 1, \quad \text{where } 0! = 1.  $\par}

\noindent \textbf{Types of Permutations:}
    \begin{enumerate}
        \item \textbf{Permutations without Repetition:} The total number of arrangements of \(n\) distinct objects taken \(r\) at a time:

          {\centering $ 
              P(n, r) = \frac{n!}{(n-r)!}.
              $\par}
            
        \item \textbf{Permutations with Repetition:} The total number of arrangements of \(n\) objects where some objects are repeated:

          {\centering $ 
              P = \frac{n!}{p_1! \cdot p_2! \cdots p_k!},
              $\par}
            
              where \(p_1, p_2, \ldots, p_k\) are the frequencies of the repeated objects.
    \end{enumerate}

% \noindent\textbf{Step-by-Step Procedure}
% \begin{enumerate}
%     \item Identify the type of permutation (with or without repetition).
%     \item For permutations without repetition:
%           \begin{enumerate}
%               \item Use the formula \(P(n, r) = \frac{n!}{(n-r)!}\).
%               \item Plug in the values of \(n\) and \(r\), and compute.
%           \end{enumerate}
%     \item For permutations with repetition:
%           \begin{enumerate}
%               \item Use the formula \(P = \frac{n!}{p_1! \cdot p_2! \cdots p_k!}\).
%               \item Identify the total objects (\(n\)) and the frequencies of repeated objects (\(p_1, p_2, \ldots, p_k\)).
%               \item Plug in the values and compute.
%           \end{enumerate}
% \end{enumerate}
