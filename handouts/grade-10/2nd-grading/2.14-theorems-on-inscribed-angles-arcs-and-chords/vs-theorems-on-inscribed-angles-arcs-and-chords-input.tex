\begin{center}
\textbf{Lesson 2.14: Theorems on Inscribed Angles, Arcs, and Chords}
\end{center}

%\vspace*{1ex}
\vspace*{-1.5ex}

\begin{itemize}
    \item An \textbf{inscribed angle} is an angle whose vertex lies on a circle and whose sides contain chords of the circle.
    \item The measure of an \textbf{inscribed angle} is half the measure of the intercepted arc.
    \item If two inscribed angles intercept the same arc, then they are congruent.
    \item A \textbf{chord} is a line segment whose endpoints lie on a circle.
    \item The perpendicular bisector of a chord passes through the center of the circle.
    \item A \textbf{tangent} is a line that touches a circle at exactly one point.
    \item The angle formed between a tangent and a chord drawn to the point of tangency is half the measure of the intercepted arc.
\end{itemize}
