\begin{center}
\textbf{Lesson 1.10.2: Finding the Factors of Polynomial Expressions of Degree 3 and Above}
\end{center}

%\vspace*{1ex}
\vspace*{-1.5ex}

\begin{itemize}
    \item A \textbf{polynomial} is an algebraic expression consisting of variables and coefficients, combined using addition, subtraction, and multiplication.
    \item Factoring polynomials of \textbf{degree 3 and above} involves various methods such as \textbf{grouping}, \textbf{synthetic division}, and the \textbf{Rational Root Theorem}.
    \item The \textbf{Factor Theorem} states that if \( f(c) = 0 \), then \( (x - c) \) is a factor of the polynomial \( f(x) \).
    \item Common techniques include:
    \begin{itemize}
        \item Factoring by \textbf{grouping}
        \item Using \textbf{synthetic division} to test possible rational roots
        \item Applying the \textbf{difference of cubes} and \textbf{sum of cubes} formulas
    \end{itemize}
\end{itemize}