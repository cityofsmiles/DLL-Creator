\begin{center}
\textbf{Lesson 2.19: Theorem on Angles Formed by Intersecting Tangents Outside a Circle}
\end{center}

%\vspace*{1ex}
\vspace*{-1.5ex}

\begin{itemize}
    \item A \textbf{tangent} to a circle is a line that touches the circle at exactly one point, called the \textbf{point of tangency}.
    \item When two tangents are drawn from an external point to a circle, they are \textbf{congruent}.
    \item The angle formed by two tangents intersecting outside a circle is given by the formula:

{\centering $
    \text{Exterior Angle} = \dfrac{1}{2} (\text{major intercepted arc} - \text{minor intercepted arc}).
$\par}
    \item This theorem helps determine unknown angle measures or arc measures in problems involving circles.
\end{itemize}
