%\vspace{1ex}
\vspace{0.3ex}
\noindent\textbf{Practice Exercises 3.10}

%\vspace{0.75ex}
\vspace{0.2ex}

Use the Midline Theorem to find the missing lengths in each of the following trapezoids.
\begin{enumerate}[label=\color{blue}\arabic*.]
    \item In trapezoid \(ABCD\), with bases \(AB = 12\) and \(CD = 20\), the midline \(EF\) joins the midpoints of \(AD\) and \(BC\). Find \(EF\).
    \item In trapezoid \(PQRS\), \(PQ = 18\) and \(MN = 24\). The midline \(MN\) joins the midpoints of \(PS\) and \(QR\). Find \(RS\).
    \item In trapezoid \(WXYZ\), the bases \(WX = 15\) and \(YZ = 25\). If the midline \(UV\) joins the midpoints of \(WY\) and \(XZ\), find \(UV\).
    \item In trapezoid \(LMNO\), \(JK = 17\) and \(NO = 24\), with midline \(JK\) connecting the midpoints of \(LO\) and \(MN\). Find \(LM\).
    \item In trapezoid \(EFGH\), where \(EF = 14\) and \(GH = 22\), the midline \(IJ\) joins the midpoints of \(EH\) and \(FG\). Find \(IJ\).
%    \item In trapezoid \(QRST\), with bases \(QR = 11\) and \(ST = 33\), the midline \(UV\) connects the midpoints of \(QT\) and \(RS\). Find \(UV\).
 %   \item In trapezoid \(ABCD\), \(AB = 16\) and \(CD = 40\), and the midline \(XY\) joins the midpoints of \(AD\) and \(BC\). Find \(XY\).
  %  \item In trapezoid \(WXYZ\), \(WX = 13\) and \(YZ = 27\), with midline \(LM\) joining the midpoints of \(WY\) and \(XZ\). Find \(LM\).
   % \item In trapezoid \(JKLM\), where \(JK = 9\) and \(LM = 29\), the midline \(PQ\) joins the midpoints of \(JM\) and \(KL\). Find \(PQ\).
    %\item In trapezoid \(OPQR\), \(OP = 8\) and \(QR = 26\), and the midline \(ST\) connects the midpoints of \(OR\) and \(PQ\). Find \(ST\).
\end{enumerate}
