\begin{center}
\textbf{Lesson 3.23: Probability of (\(\mathbf{A \cup B}\)) Using a Venn Diagram}
\end{center}

%\vspace*{1ex}
\vspace*{-1.5ex}

\noindent The probability of the union of two events \( A \cup B \) is given by the formula:

{\centering $ 
P(A \cup B) = P(A) + P(B) - P(A \cap B)
 $\par}

\noindent where:
\begin{itemize}
    \item \( P(A) \) is the probability of event \( A \),
    \item \( P(B) \) is the probability of event \( B \),
    \item \( P(A \cap B) \) is the probability of both events occurring simultaneously% (the intersection of \( A \) and \( B \)).
\end{itemize}

In a Venn diagram, the probability of \( A \cup B \) is represented by the area covered by both circles \( A \) and \( B \).

% \noindent\textbf{Step-by-Step Procedure:}

% \begin{enumerate}
%     \item Identify the total sample space and the events \( A \) and \( B \).
%     \item Find the probabilities of events \( A \), \( B \), and \( A \cap B \).
%     \item Use the formula to calculate the probability of the union.
%     \item If given a Venn diagram, find the areas corresponding to \( A \), \( B \), and \( A \cap B \) and apply the formula.
% \end{enumerate}
