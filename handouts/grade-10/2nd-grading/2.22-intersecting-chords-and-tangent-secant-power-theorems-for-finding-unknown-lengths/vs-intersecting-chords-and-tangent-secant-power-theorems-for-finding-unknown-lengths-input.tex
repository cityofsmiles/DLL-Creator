\begin{center}
\textbf{Lesson 2.22: Intersecting Chords and Tangent-Secant Power Theorems for Finding Unknown Lengths}
\end{center}

%\vspace*{1ex}
\vspace*{-1.5ex}

\begin{itemize}
    \item The \textbf{Intersecting Segments of Chords Theorem} states that if two chords intersect inside a circle, the products of the lengths of their segments are equal:

{\centering $
    AP \cdot PB = CP \cdot PD
$\par}
    where $A, B, C,$ and $D$ are points on the circle, and $P$ is the intersection of the chords.
    
    \item The \textbf{Tangent-Secant Power Theorem} states that if a tangent and a secant are drawn from the same external point, then the square of the length of the tangent is equal to the product of the external and total length of the secant:

{\centering $
    (PT)^2 = PQ \cdot PR
$\par}
    where $PT$ is the tangent, and $PQ$ and $PR$ are the external and total parts of the secant.
    
    \item These theorems help find unknown segment lengths in problems involving circles.
\end{itemize}
