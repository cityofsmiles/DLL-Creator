\begin{center}
\textbf{Lesson 4.02: Variables and Constants in Algebraic Expressions}
\end{center}

%\vspace*{1ex}
\vspace*{-1.5ex}

\noindent \textbf{Variable:} a symbol (usually a letter) that represents an unknown or changeable value. Examples: $x$, $y$, $z$

\noindent \textbf{Constant:} a fixed value that does not change. Examples: $5$, $-3$, $\frac{1}{2}$

\noindent \textbf{Steps to Distinguish Variables and Constants}
\begin{enumerate}
    \item Look for symbols in the expression. Letters are usually variables.  
    \item Identify the fixed numerical values. These are constants.  
    \item Pay attention to the context. Some expressions may have implicit constants (e.g., $-4$ in $3x - 4$).  
    \item Write the variables and constants separately for clarity.  
\end{enumerate}
