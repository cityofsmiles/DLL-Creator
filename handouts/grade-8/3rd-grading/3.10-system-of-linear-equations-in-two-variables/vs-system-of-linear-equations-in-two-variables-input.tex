 \begin{center}
\textbf{Lesson 3.10: System of Linear Equations in Two Variables}
\end{center}

\vspace*{-2ex}

    \noindent \textbf{System of Linear Equations}: Two or more linear equations considered simultaneously.

    \noindent \textbf{Solution of a System}: The set of ordered pairs $(x, y)$ that satisfy both equations in the system. It represents the point(s) where the lines intersect.

    \noindent \textbf{Consistent System}: A system that has at least one solution.

    \noindent \textbf{Inconsistent System}: A system that has no solutions.

    \noindent \textbf{Dependent System}: A system where the equations represent the same line (infinitely many solutions).

    \noindent \textbf{Independent System}: A system where the equations represent different lines that intersect at a single point (one solution).


\noindent\textbf{Steps to Illustrate a System of Linear Equations}

\begin{enumerate}
    \item \textbf{Write the two equations in slope-intercept form} ($y = mx + b$), if necessary.
    \item \textbf{Graph each equation} on the same coordinate plane.
    \item \textbf{Identify the point of intersection}, if there is one.
    \item \textbf{Determine the solution}: 
    \begin{itemize}
        \item If the lines intersect at one point, the system is \textbf{consistent and independent}.
        \item If the lines are parallel, the system is \textbf{inconsistent}.
        \item If the lines coincide, the system is \textbf{consistent and dependent}.
    \end{itemize}
\end{enumerate}

