\begin{center}
\textbf{Lesson 3.09: Combinations of Objects}
\end{center}

%\vspace*{1ex}
\vspace*{-1.5ex}

\noindent\textbf{Combination:} a selection of \(r\) objects from \(n\) objects where the order does not matter such as forming committees, choosing teams, or selecting items

\noindent The formula for finding the number of combinations is:

{\centering $ 
    C(n, r) = \binom{n}{r} = \dfrac{n!}{r!(n-r)!}
 $\par}

\noindent where \(n\) is the total number of objects, \(r\) is the number of objects chosen, and \(n! = n \cdot (n-1) \cdot (n-2) \cdots 1\) is the factorial of \(n\).

\noindent\textbf{Step-by-Step Procedure}

\begin{enumerate}
    \item Identify the values of \(n\) (total objects) and \(r\) (objects to select).
    \item Use the formula:    \(C(n, r) = \dfrac{n!}{r!(n-r)!}\)
    \item Compute the factorials of \(n\), \(r\), and \((n-r)\).
    \item Simplify the expression to find the total number of combinations.
\end{enumerate}
