\begin{center}
\textbf{Lesson 2.11: Illustrating Sets, Subsets, Unions, and Intersections of Sets Using Venn Diagrams}
\end{center}

%\vspace*{1ex}
\vspace*{-1.5ex}

\noindent\textbf{Union of \(A\) and \(B\)} (\(A \cup B\)): is the set whose elements belong to set \(A\) or set \(B\), or both

\noindent\textbf{Intersection of \(A\) and \(B\)} (\(A \cap B\)): the set whose elements are the common elements of set \(A\) and set \(B\) 

\noindent\textbf{Joint Sets:} if the intersection of two sets is not an empty set

\noindent\textbf{Disjoint Sets:} if the intersection of two sets is a null set

\noindent Let \(A\) be any set and \(U\) be the universal set. Then \(A \cap A = A\), \(A \cap U = A\), and \(A \cap \emptyset = \emptyset\).

\noindent\textbf{Set Difference of \(B\) from \(A\)} (\(A - B\)): the set of all elements in set \(A\) that are not elements of set \(B\)

\noindent\textbf{Complement of Set A} (\(A'\)): the set containing the elements in \(U\) that do not belong to set \(A\)

\noindent\textbf{Venn Diagram:} a visual representation that uses rectangles and circles to represent sets



