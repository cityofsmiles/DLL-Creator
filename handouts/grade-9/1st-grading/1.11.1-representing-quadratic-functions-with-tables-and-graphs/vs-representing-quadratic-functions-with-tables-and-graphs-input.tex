\begin{center}
\textbf{Lesson 1.11.1: Representing Quadratic Functions with Tables and Graphs}
\end{center}

\vspace*{1ex}

\textbf{Equal Differences Method:} A function represents a table of values having the differences of the $y$ values. If the second differences of the $y$ values are equal, then the given function is a quadratic function. 

\textbf{Representing Quadratic Functions with Tables}

To represent a quadratic function $f(x) = ax^2 + bx + c$ with a table:
\begin{enumerate}[label = \color{blue}\arabic*. ]
    \item Choose a range of values for $x$.
    \item Compute the corresponding values of $f(x)$ using the quadratic function.
    \item Organize the values in a table with two columns: one for $x$ and one for $f(x)$.
\end{enumerate}

\textbf{Representing Quadratic Functions with Graphs}

The graph of a quadratic function is called a \textbf{parabola}. It is basically a curved shape opening up or down. 

When you have a quadratic function in form $f(x) = ax^2 + bx + c$, if $a > 0$, then the parabola opens upward; if $a < 0$, then the parabola opens down. 

To represent a quadratic function graphically:
\begin{enumerate}[label = \color{blue}\arabic*. ]
    \item Plot the vertex of the parabola, which is given by the point $(h, k)$, where $h = -\frac{b}{2a}$ and $k = f(h)$.
    \item Plot additional points by choosing values of $x$ and computing the corresponding values of $f(x)$.
    \item Connect the points to form the graph of the quadratic function.
\end{enumerate}


