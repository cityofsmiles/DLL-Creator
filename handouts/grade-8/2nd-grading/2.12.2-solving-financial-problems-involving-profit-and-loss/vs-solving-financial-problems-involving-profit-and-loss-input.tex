 \begin{center}
\textbf{Lesson 2.12.2: Solving Financial Problems Involving Profit and Loss}
\end{center}

%\vspace*{1ex}

 \noindent   \textbf{Cost Price (CP):} The price at which an item is purchased.\\
 \noindent    \textbf{Selling Price (SP):} The price at which an item is sold.

  \noindent   \textbf{Profit:} When the selling price is higher than the cost price.
  
{\centering $\text{Profit} = \text{SP} - \text{CP}$\par}

\noindent \textbf{Loss:} When the selling price is lower than the cost price.

{\centering \text{Loss} = \text{CP} - \text{SP}\par}

\noindent \textbf{Profit Percentage:} The percentage of profit made on the cost price.

{\centering $ \text{Profit Percentage} = \left( \dfrac{\text{Profit}}{\text{CP}} \right) \times 100 $\par}

\noindent \textbf{Loss Percentage:} The percentage of loss incurred on the cost price.

{\centering $ \text{Loss Percentage} = \left( \dfrac{\text{Loss}}{\text{CP}} \right) \times 100 $\par}

