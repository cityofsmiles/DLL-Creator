%\vspace{1ex}
\vspace{0.3ex}
\noindent\textbf{Practice Exercises 3.14}

%\vspace{0.75ex}
\vspace{0.2ex}

Use the properties of trapezoids to solve each problem.

\begin{enumerate}[label=\color{blue}\arabic*.]
    \item In trapezoid \(ABCD\), \(AB\) and \(CD\) are parallel with \(AB = 12\) cm and \(CD = 20\) cm. Find the length of the midsegment.
    %\item In isosceles trapezoid \(EFGH\), \(EF = 14\) cm, \(GH = 26\) cm, and the height is 8 cm. Find the area of \(EFGH\).
    \item In trapezoid \(PQRS\), \(PQ = (3x - 5)\) and \(RS = 24\) cm. If the midsegment is 17 cm, find \(x\).
    \item In trapezoid \(JKLM\), \(JK\) and \(LM\) are parallel, and the length of the midsegment is 22 cm. If \(JK = 18\) cm, find \(LM\).
   % \item In trapezoid \(WXYZ\), \(WX = 15\) cm, \(YZ = 25\) cm, and the height is 10 cm. Calculate the area of \(WXYZ\).
    \item In trapezoid \(ABCD\), the bases \(AB = (2a + 10)\) cm and \(CD = 30\) cm. If the midsegment is 25 cm, find \(a\).
  %  \item In isosceles trapezoid \(RSTU\), \(RS = 16\) cm and \(TU = 34\) cm. The height of the trapezoid is 12 cm. Calculate the area.
    \item In trapezoid \(EFGH\), \(EF\) and \(GH\) are parallel, and the midsegment is 18 cm. If \(EF = 20\) cm, find \(GH\).
 %   \item In trapezoid \(JKLM\), if \(JK = (2b + 6)\) cm and \(LM = 28\) cm, and the midsegment is 19 cm, find \(b\).
 %   \item In isosceles trapezoid \(PQRS\), \(PQ = 10\) cm, \(RS = 30\) cm, and the height is 15 cm. Find the area of \(PQRS\).
\end{enumerate}
