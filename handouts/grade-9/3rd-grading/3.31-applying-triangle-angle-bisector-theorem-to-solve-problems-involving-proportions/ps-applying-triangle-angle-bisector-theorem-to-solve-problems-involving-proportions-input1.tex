%\vspace{1ex}
\vspace{0.3ex}
\noindent\textbf{Practice Exercises 3.31}

%\vspace{0.75ex}
\vspace{0.2ex}

Use the Triangle Angle Bisector Theorem to solve the following problems.

\begin{enumerate}[label=\color{blue}\arabic*.]
    \item In \(\triangle ABC\), \(\overline{AD}\) is the angle bisector of \(\angle BAC\). If \(AB = 8\), \(AC = 12\), and \(BD = 6\), find \(DC\).
    \item In \(\triangle XYZ\), \(\overline{XP}\) is the angle bisector of \(\angle YXZ\). If \(XY = 9\), \(XZ = 12\), and \(YP = 7.5\), find \(PZ\).
    \item In \(\triangle DEF\), \(\overline{DM}\) is the angle bisector of \(\angle EDF\). If \(DE = 10\), \(DF = 15\), and \(EM = 4\), find \(MF\).
    \item In \(\triangle PQR\), \(\overline{QT}\) is the angle bisector of \(\angle PQR\). If \(PQ = 14\), \(QR = 21\), and \(PT = 10\), find \(TR\).
    \item In \(\triangle ABC\), \(\overline{AE}\) is the angle bisector of \(\angle BAC\). If \(AB = 6\), \(AC = 9\), and \(BE = 5\), find \(EC\).
    % \item In \(\triangle XYZ\), \(\overline{XP}\) is the angle bisector of \(\angle YXZ\). If \(XY = 8\), \(XZ = 10\), and \(YP = 4\), find \(PZ\).
    % \item In \(\triangle DEF\), \(\overline{DM}\) is the angle bisector of \(\angle EDF\). If \(DE = 12\), \(DF = 18\), and \(EM = 5\), find \(MF\).
    % \item In \(\triangle PQR\), \(\overline{QT}\) is the angle bisector of \(\angle PQR\). If \(PQ = 16\), \(QR = 24\), and \(PT = 6\), find \(TR\).
    % \item In \(\triangle ABC\), \(\overline{AE}\) is the angle bisector of \(\angle BAC\). If \(AB = 7\), \(AC = 14\), and \(BE = 3.5\), find \(EC\).
    % \item In \(\triangle XYZ\), \(\overline{XP}\) is the angle bisector of \(\angle YXZ\). If \(XY = 10\), \(XZ = 15\), and \(YP = 6\), find \(PZ\).
\end{enumerate}
