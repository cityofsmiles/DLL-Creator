\begin{center}
\textbf{Lesson 1.10: Identifying and Explaining the Uses of Rates}
\end{center}

%\vspace*{1ex}
\vspace*{-1.5ex}

\begin{itemize}
    \item A \textbf{rate} is a comparison of two quantities with different units.
    \item Common examples of rates include:
    \begin{itemize}
        \item Speed (e.g., kilometers per hour)
        \item Price (e.g., pesos per kilogram)
        \item Work efficiency (e.g., tasks completed per hour)
        \item Consumption (e.g., liters per minute)
    \end{itemize}
    \item A \textbf{unit rate} is a rate where the denominator is 1.
    \item Rates are useful in everyday life for:
    \begin{itemize}
        \item Comparing prices and making purchasing decisions
        \item Measuring speed and fuel efficiency
        \item Calculating wages and salaries
        \item Understanding health statistics (e.g., heartbeats per minute)
    \end{itemize}
\end{itemize}
