\begin{center}
\textbf{Lesson 1.09.3: Applying the Factor Theorem and Remainder Theorem to Solve Problems}
\end{center}

%\vspace*{1ex}
\vspace*{-1.5ex}

\begin{itemize}
    \item The \textbf{Remainder Theorem} states that if a polynomial \( f(x) \) is divided by \( (x - c) \), the remainder is \( f(c) \).
    \item The \textbf{Factor Theorem} states that \( (x - c) \) is a factor of \( f(x) \) if and only if \( f(c) = 0 \).
    \item These theorems are useful in determining unknown coefficients in polynomials and checking divisibility.
    \item To find an unknown \( k \), substitute given values into the polynomial equation and solve.
    \item Both theorems provide a quick way to verify factors without performing polynomial division.
\end{itemize}
