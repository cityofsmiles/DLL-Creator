\begin{center}
\textbf{Lesson 2.22: Simplifying Radical Expressions by Rationalizing the Denominator}
\end{center}

\vspace*{1ex}
%\vspace*{-1.5ex}

A radical is in \textbf{simplified form} if all of the following are true: 

%\begin{multicols}{2}
\setlist{nolistsep}
\begin{enumerate}[noitemsep, label = \color{blue}\arabic*. ]
\item There are no factors in the radicand that can be written as powers greater than or equal to the index. 
\item There no fractions under the radical sign.
\item There are no radicals in the denominator.
\end{enumerate}
% \end{multicols}


\textbf{Rationalizing the denominator} involves eliminating radicals from the denominator of a fraction by multiplying both the numerator and the denominator by an appropriate expression.

In case of a binomial in the denominator involving square roots, use \textbf{conjugates}.

   
