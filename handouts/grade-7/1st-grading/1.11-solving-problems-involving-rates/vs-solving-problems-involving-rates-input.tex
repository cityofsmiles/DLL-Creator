\begin{center}
\textbf{Lesson 1.11: Solving Problems Involving Rates}
\end{center}

%\vspace*{1ex}
\vspace*{-1.5ex}

\begin{itemize}
    \item A \textbf{rate} is a comparison of two quantities with different units, such as kilometers per hour (km/h) or pesos per kilogram.
    \item \textbf{Speed} is a common type of rate that measures how fast an object moves. It is calculated as: \hfil $    \text{Speed} = \dfrac{\text{Distance}}{\text{Time}} $ \hfil\par

%{\centering $    \text{Speed} = \dfrac{\text{Distance}}{\text{Time}} $\par}

    \item The formula can be rearranged to solve for distance or time:

{\centering $    \text{Distance} = \text{Speed} \times \text{Time} $ \quad \quad $    \text{Time} = \dfrac{\text{Distance}}{\text{Speed}} $\par}

%{\centering $    \text{Time} = \dfrac{\text{Distance}}{\text{Speed}} $\par}

    \item Other examples of rates include:
      \begin{itemize}
        \begin{multicols}{2}
        \item Work rate: tasks completed per hour
        \item Fuel efficiency: kilometers per liter
        \item Flow rate: liters per second
          \end{multicols}
    \end{itemize}
 %   \item Rates are useful in everyday life for travel planning, budgeting, and resource management.
\end{itemize}
