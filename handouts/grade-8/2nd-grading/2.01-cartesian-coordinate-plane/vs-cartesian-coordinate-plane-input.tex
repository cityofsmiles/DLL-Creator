 \begin{center}
\textbf{Lesson 2.1: Cartesian Coordinate Plane}
\end{center}

\vspace*{1ex}


\textbf{Cartesian Coordinate System:} constructed by drawing two perpendicular lines wherein the point of intersection is called the \emph{origin  }

Coordinate axes: the two perpendicular lines

X-Axis: the horizontal line

Y-Axis: the vertical line

Quadrants: the four regions that divide the plane

Coordinates: the ordered pair of real numbers that corresponds to each point in the plane
 
X-Coordinate or Abscissa: the first number of the ordered pair

Y-Coordinate or Ordinate: the second  number of the ordered pair


The origin (O) is the point \((0, 0)\) where the x-axis and y-axis intersect. The plane is divided into four quadrants:
\begin{enumerate}
    \item \textbf{Quadrant I}: Both \(x\) and \(y\) are positive.
    \item \textbf{Quadrant II}: \(x\) is negative, and \(y\) is positive.
    \item \textbf{Quadrant III}: Both \(x\) and \(y\) are negative.
    \item \textbf{Quadrant IV}: \(x\) is positive, and \(y\) is negative.
\end{enumerate}
