\begin{center}
\textbf{Lesson 1.12.1: Converting Quadratic Functions to Vertex Form}
\end{center}

\vspace*{1ex}

The vertex form or standard form of a quadratic function is:
\[ f(x) = a(x-h)^2 + k \]
where \((h, k)\) represents the vertex of the parabola, making it easy to identify the maximum or minimum value of the function.

\textbf{Steps to Convert from General Form to Vertex Form}
%\begin{multicols}{2}
\begin{enumerate}[label = \color{blue}\arabic*. ]
%1
\item Extract the Coefficient: Start with the general form. 
Extract \( a \) if it is not 1. 
%2
\item Complete the square inside the parentheses. 
%3
\item Simplify the expression to achieve the vertex form. 
\end{enumerate}
%\end{multicols} 


