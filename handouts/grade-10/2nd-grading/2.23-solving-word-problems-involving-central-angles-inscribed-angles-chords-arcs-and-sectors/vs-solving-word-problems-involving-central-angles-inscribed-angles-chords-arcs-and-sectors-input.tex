\begin{center}
\textbf{Lesson 2.23: Solving Word Problems Involving Central Angles, Inscribed Angles, Chords, Arcs, and Sectors}
\end{center}

%\vspace*{1ex}
\vspace*{-1.5ex}

\begin{itemize}
    \item A \textbf{central angle} is an angle whose vertex is at the center of the circle, and its measure is equal to the measure of the intercepted arc.
    \item An \textbf{inscribed angle} is an angle whose vertex is on the circle, and its measure is half the measure of the intercepted arc:

{\centering $
    \text{Inscribed Angle} = \dfrac{1}{2} \times \text{Intercepted Arc}
$\par}
    \item A \textbf{chord} is a line segment whose endpoints lie on the circle. The perpendicular bisector of a chord passes through the center of the circle.
    \item An \textbf{arc} is a portion of the circumference of a circle. The measure of a minor arc is equal to its corresponding central angle.
    \item The \textbf{sector} of a circle is a region bounded by two radii and an arc. The area of a sector is given by:

{\centering $
    A = \pi r^2 \times \dfrac{\theta}{360^\circ}
$\par}
    where $r$ is the radius and $\theta$ is the central angle in degrees.
\end{itemize}
