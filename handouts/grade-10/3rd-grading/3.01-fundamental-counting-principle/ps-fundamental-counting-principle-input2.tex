%\vspace{1ex}
\vspace{0.3ex}
\noindent\textbf{Activity 3.01}

%\vspace{0.75ex}
\vspace{0.2ex}

Use the Fundamental Counting Principle to find the number of possible arrangements for each scenario.

\begin{enumerate}[label=\color{blue}\arabic*.]
    % \item A password consists of 2 letters followed by 3 digits. How many different passwords can be created if there are 26 letters and 10 digits available?
    % \item A pizza place offers 2 sizes, 5 crust types, and 6 topping options. How many different types of pizza can be made?
    % \item A restaurant offers 4 types of drinks, 6 appetizers, and 3 entrees. How many different meal combinations can a customer choose?
    \item A combination lock uses 4 numbers, each between 0 and 9. How many different combinations are possible?
    % \item A class has 6 different books, and a student needs to select 3 books to read. How many different ways can the student choose 3 books?
    \item A concert hall has 4 different sections, 12 rows in each section, and 20 seats in each row. How many total seats are there in the hall?
    \item A team has 7 forwards, 5 midfielders, and 3 defenders. How many ways can a coach select one forward, one midfielder, and one defender to start the game?
    \item A student needs to choose a shirt (3 options), a pair of pants (4 options), and shoes (5 options). How many different outfits can the student wear?
    \item A license plate consists of 3 letters followed by 4 digits. How many different license plates can be made if there are 26 letters and 10 digits available?
%    \item A group of 4 friends wants to sit in a row at a movie theater with 10 available seats. How many different seating arrangements are possible?
\end{enumerate}
