\begin{center}
\textbf{Lesson 1.6: Multiplying Monomials and Binomials Using the Distributive Property}
\end{center}

%\vspace*{1ex}

    \textbf{Binomial:} An algebraic expression with two terms. Examples include \( x + 2 \), \( 3y - 5 \), and \( 2a + 4b \).

    \textbf{Distributive Property:} A property that allows you to multiply a single term by each term within parentheses. For example, \( 3(x + 4) = 3x + 12 \).

\textbf{Multiplying Monomials with Binomials:} Use the distributive property to multiply the monomial by each term in the binomial. For example, \( 2x(3x + 4) = 6x^2 + 8x \).

\textbf{Multiplying Binomials with Binomials:} Apply the distributive property twice, or use the FOIL method (First, Outer, Inner, Last) to multiply the terms. For example, \( (x + 3)(x + 2) = x^2 + 2x + 3x + 6 = x^2 + 5x + 6 \).

\textbf{Multiplying Binomials with Multinomials:} Use the distributive property multiple times to multiply each term in the binomial by each term in the multinomial.

 

 

