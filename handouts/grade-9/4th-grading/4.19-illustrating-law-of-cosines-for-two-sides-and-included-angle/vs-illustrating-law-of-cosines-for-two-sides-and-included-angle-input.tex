\begin{center}
\textbf{Lesson 4.19: Illustrating Law of Cosines for Two Sides and Included Angle}
\end{center}

%\vspace*{1ex}
\vspace*{-1.5ex}

The \textbf{Law of Cosines} allows us to solve triangles when we know two sides and the included angle. It is especially useful when working with oblique triangles. The formula for the Law of Cosines is:  

{\centering $
c^2 = a^2 + b^2 - 2ab \cos C
$\par}

\noindent where:
\begin{itemize}
    \item \(a\) and \(b\) are the lengths of the two known sides.
    \item \(C\) is the included angle (the angle between sides \(a\) and \(b\)).
    \item \(c\) is the side opposite angle \(C\), which we aim to find.
\end{itemize}

\noindent\textbf{Step-by-Step Procedure:}
\begin{enumerate}
    \item Identify the known values. Note the lengths of the two sides (\(a\) and \(b\)) and the measure of the included angle (\(C\)).
    \item Write the Law of Cosines formula. 
    \item Substitute the known values. Plug in the lengths of \(a\), \(b\), and \(\cos C\).
    \item Calculate \(c^2\). Perform the necessary operations.
    \item Solve for \(c\). Take the square root of \(c^2\) to find \(c\).
    \item Check your answer. Ensure is reasonable within the context of the triangle.
\end{enumerate}
