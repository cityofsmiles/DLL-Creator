 \begin{center}
\textbf{Lesson 4.2: Drawing Conclusions from Statistical Data Using Measures of Variability}
\end{center}

\vspace*{1ex}

\documentclass[12pt]{article}
\usepackage{amsmath}

\title{Grade 8 Math: Drawing Conclusions from Statistical Data Using Measures of Variability}
\author{Teacher's Name}
\date{\today}

\begin{document}

\maketitle

\section*{Lesson Overview}

In this lesson, students will focus on interpreting statistical data using measures of variability, such as range, mean deviation, and standard deviation. These measures will help students understand how spread out the data is and draw conclusions about the consistency, reliability, or trends in the data set.

\section*{Key Terms}

\begin{itemize}
    \item \textbf{Range}: The difference between the highest and lowest data values.
    \item \textbf{Mean Deviation}: The average of the absolute differences between each data value and the mean.
    \item \textbf{Standard Deviation}: A measure of how much the data values deviate from the mean, calculated as the square root of the variance.
    \item \textbf{Interpretation}: The process of drawing conclusions about data based on its variability, determining if the data is consistent, spread out, or shows any particular trend.
\end{itemize}

\section*{Steps for Interpretation of Data Using Variability Measures}

\begin{enumerate}
    \item \textbf{Examine the range}: A large range suggests a big difference between the smallest and largest values, indicating greater variability.
    \item \textbf{Check the mean deviation}: A high mean deviation shows that most data points are far from the average, indicating inconsistency in the data.
    \item \textbf{Interpret the standard deviation}: A high standard deviation means data points are more spread out from the mean, while a low standard deviation suggests the data points are clustered around the mean.
    \item \textbf{Draw conclusions}: Based on the variability, determine if the data shows a trend of consistency, high variation, or if it suggests any specific patterns.
\end{enumerate}

\section*{Practice Exercises}

For the following scenarios, interpret the data using the range, mean deviation, and standard deviation provided. Answer the questions that follow based on your interpretation of the results.

\begin{enumerate}
    \item A basketball player’s scores in 10 games have a range of 35 points, a mean deviation of 4.5, and a standard deviation of 7. What can you conclude about the player’s consistency in scoring?
    \item A survey was conducted on daily screen time among students. The data has a range of 3 hours, a mean deviation of 0.5 hours, and a standard deviation of 1. What can you say about the variation in students' daily screen time?
    \item The heights of plants in a greenhouse have a range of 15 cm, a mean deviation of 2 cm, and a standard deviation of 3.5 cm. What do these measures suggest about the growth of plants?
    \item A test was administered to two different classes. Class A had a standard deviation of 2, while Class B had a standard deviation of 5. Which class had more variation in test scores? What does this say about their performance?
    \item In a study of household electricity usage, the data shows a range of 200 kWh, a mean deviation of 15 kWh, and a standard deviation of 20 kWh. What conclusion can you draw about how different households use electricity?
    \item A company tracks the weekly sales of its top product. The range is 500 units, the mean deviation is 35 units, and the standard deviation is 45 units. What do these results suggest about the consistency of the product’s sales?
    \item Two students measured the temperature every hour for 24 hours. Student A has a standard deviation of 0.5°C, while Student B has a standard deviation of 3°C. Whose data shows more temperature variation? What does this tell us about their measurements?
    \item A teacher graded a set of 30 exams and found that the range of scores was 20 points, with a mean deviation of 2.5 points and a standard deviation of 4. How would you interpret the variability in the students’ exam scores?
    \item A scientist records the daily rainfall in two cities over a month. City A has a range of 50 mm and a standard deviation of 8 mm, while City B has a range of 30 mm and a standard deviation of 4 mm. What conclusions can you draw about the rainfall in each city?
    \item A store measures customer spending patterns. The data has a range of \$200, a mean deviation of \$20, and a standard deviation of \$25. What can you conclude about how much customers typically spend?
\end{enumerate}

\section*{Activities}

For the following real-world scenarios, interpret the data using the range, mean deviation, and standard deviation provided. Answer the questions based on your interpretation of the variability measures.

\begin{enumerate}
    \item A runner tracks their times for 10 races. The range is 20 seconds, the mean deviation is 3.5 seconds, and the standard deviation is 5 seconds. What does this suggest about the runner’s consistency?
    \item The number of books read by students in a class over a semester has a range of 8 books, a mean deviation of 1.2 books, and a standard deviation of 2 books. What can you conclude about the students’ reading habits?
    \item A company’s profits over the past 6 months have a range of \$50,000, a mean deviation of \$5,000, and a standard deviation of \$7,000. What does this tell you about the company’s financial stability?
    \item Two employees track their daily steps for 7 days. Employee A has a standard deviation of 200 steps, while Employee B has a standard deviation of 500 steps. Which employee’s activity shows more variability?
    \item A sample of exam scores from two schools shows that School A has a range of 30 points, a mean deviation of 4 points, and a standard deviation of 5 points. School B has a range of 25 points, a mean deviation of 3 points, and a standard deviation of 4 points. Which school has more consistent scores?
    \item A store’s daily revenue has a range of \$1,000, a mean deviation of \$120, and a standard deviation of \$150. What conclusions can you draw about the store’s revenue?
    \item Two farmers track the growth of crops in their fields. Farmer A’s data shows a range of 10 cm, a mean deviation of 1.5 cm, and a standard deviation of 2 cm. Farmer B’s data shows a range of 8 cm, a mean deviation of 1 cm, and a standard deviation of 1.5 cm. Which farmer has more consistent crop growth?
    \item A teacher evaluates students' homework scores over a month. The range is 5 points, the mean deviation is 0.8 points, and the standard deviation is 1.2 points. What does this say about the students' performance?
    \item Two friends track their time spent on social media. Friend A has a range of 4 hours, a mean deviation of 0.5 hours, and a standard deviation of 1 hour. Friend B has a range of 2 hours, a mean deviation of 0.3 hours, and a standard deviation of 0.6 hours. What can you conclude about their usage habits?
    \item A company's stock prices over a month show a range of \$10, a mean deviation of \$1.5, and a standard deviation of \$2. What does this suggest about the stock’s volatility?
\end{enumerate}


