\begin{center}
\textbf{Lesson 4.18: Illustrating Law of Sines for Two Sides and an Angle Opposite One Side Given}
\end{center}

%\vspace*{1ex}
\vspace*{-1.5ex}

\noindent\textbf{Law of Sines:}  
For any triangle, the Law of Sines states:  

{\centering $
\dfrac{a}{\sin A} = \dfrac{b}{\sin B} = \dfrac{c}{\sin C},
$\par}

\noindent where \(a\), \(b\), and \(c\) are the sides opposite to angles \(A\), \(B\), and \(C\), respectively.

\noindent\textbf{Procedure:}  
\begin{enumerate}
    \item Identify the given sides and angle of the triangle.  
    \item Use the Law of Sines formula: \(\dfrac{a}{\sin A} = \dfrac{b}{\sin B}\) to solve for the missing angle.  
    \item Use the triangle's angle sum property (\(A + B + C = 180^\circ\)) to find the remaining angle.  
    \item Use the Law of Sines to solve for the remaining side.  
    \item Simplify the calculations and ensure units are consistent.
\end{enumerate}
