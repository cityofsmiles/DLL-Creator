\begin{center}
\textbf{Lesson 1.09.2: Using the Factor Theorem to Determine Factors of a Polynomial}
\end{center}

%\vspace*{1ex}
\vspace*{-1.5ex}

\begin{itemize}
    \item The \textbf{Factor Theorem} states that a polynomial \( P(x) \) has \( (x - c) \) as a factor if and only if \( P(c) = 0 \).
    \item If substituting \( x = c \) into the polynomial results in zero, then \( (x - c) \) is a factor.
    \item This theorem provides a straightforward method for checking divisibility without performing polynomial division.
    \item The theorem is useful in factoring polynomials completely and finding their roots.
\end{itemize}