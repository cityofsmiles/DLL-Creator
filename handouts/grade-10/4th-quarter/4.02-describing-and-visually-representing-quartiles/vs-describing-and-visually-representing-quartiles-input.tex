\begin{center}
\textbf{Lesson 4.02: Describing and Visually Representing Quartiles}
\end{center}

%\vspace*{1ex}
\vspace*{-1.5ex}

\noindent \textbf{Quartiles:} Quartiles divide a data set into four equal parts.  
    \begin{itemize}
        \item The first quartile (\(Q_1\)) is the median of the lower half of the data.  
        \item The second quartile (\(Q_2\)) is the median of the entire data set.  
        \item The third quartile (\(Q_3\)) is the median of the upper half of the data.  
    \end{itemize}

\noindent \textbf{Interquartile Range (IQR):} The range of the middle 50\% of the data, calculated as \(Q_3 - Q_1\).

\noindent\textbf{Procedure:}  
\begin{enumerate}
    \item Arrange the data in ascending order.  
    \item Identify the median (\(Q_2\)).  
    \item Divide the data into two halves.
    % \begin{itemize}
    %     \item The lower half consists of all values below \(Q_2\).  
    %     \item The upper half consists of all values above \(Q_2\).  
    % \end{itemize}
    \item Find the median of the lower half (\(Q_1\)).  
    \item Find the median of the upper half (\(Q_3\)).
%\item Represent the quartiles visually using a box-and-whisker plot. 
\end{enumerate}
