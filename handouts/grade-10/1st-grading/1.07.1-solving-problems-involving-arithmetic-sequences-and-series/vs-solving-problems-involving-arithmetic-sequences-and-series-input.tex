\begin{center}
\textbf{Lesson 1.07.1: Solving Problems Involving Arithmetic Sequences and Series}
\end{center}

%\vspace*{1ex}
\vspace*{-1.5ex}

\begin{itemize}
    \item An \textbf{arithmetic sequence} is a sequence in which the difference between consecutive terms is constant, called the \textbf{common difference} (\(d\)).
    % \item The \textbf{nth term} of an arithmetic sequence is given by:
    %       \[ a_n = a_1 + (n-1)d \]
    %       where:
    %       \begin{itemize}
    %           \item \( a_n \) is the nth term,
    %           \item \( a_1 \) is the first term, and
    %           \item \( d \) is the common difference.
    %       \end{itemize}
    % \item The \textbf{sum} of the first \(n\) terms of an arithmetic sequence (arithmetic series) is given by:
    %       \[ S_n = \dfrac{n}{2} (2a_1 + (n-1)d) \]
    %       or equivalently,
    %       \[ S_n = \dfrac{n}{2} (a_1 + a_n) \]
    \item Arithmetic sequences and series are used in solving real-life problems, such as calculating savings, salary increments, and number patterns in nature.
\end{itemize}
