\begin{center}
\textbf{Lesson 3.01: Data Collection and Sampling Techniques}
\end{center}

%\vspace*{1ex}
\vspace*{-1.5ex}

\noindent\textbf{Statistics:} the study of collecting, organizing, and analyzing data

\noindent\textbf{Data:} information used to analyze something or to make decisions

\noindent\textbf{Types of Data}
%\begin{multicols}{2}
\setlist{nolistsep}
\begin{enumerate}[noitemsep, label = \color{blue}\arabic*. ]
\item Quantitative data: can be represented by numbers
\item Qualitative data: information that can be grouped into different categories
\end{enumerate}
% \end{multicols}

\noindent\textbf{Sample:} a smaller group taken from a population

\noindent\textbf{Sampling Techniques}
%\begin{multicols}{2}
\setlist{nolistsep}
\begin{enumerate}[noitemsep, label = \color{blue}\arabic*. ]
\item Simple Random Sampling: everyone in the population has an equal chance of being chosen to be part of the sample
\item Stratified Random Sampling: divides the big population into smaller groups, called \textbf{strata}
\item Cluster Random Sampling: the sample consists of elements of a specific group
\item Systematic Random Sampling: the sample is created by choosing members systematically from an ordered list of the population
\end{enumerate}
% \end{multicols}

\noindent\textbf{Data Collection Methods}
%\begin{multicols}{2}
\setlist{nolistsep}
\begin{enumerate}[noitemsep, label = \color{blue}\arabic*. ]
\item Actual measurement
\item Interview
\item Survey or questionnaire
\item Observation: can be direct or indirect
\end{enumerate}
% \end{multicols}
