\begin{center}
\textbf{Lesson 2.06: Graphing Polynomial Functions with Even Degree and Negative Leading Coefficient}
\end{center}

%\vspace*{1ex}
\vspace*{-1.5ex}

\begin{itemize}
    \item A \textbf{polynomial function} has the form:

 {\centering $
    P(x) = a_n x^n + a_{n-1} x^{n-1} + \dots + a_1 x + a_0
    $\par}
  
    where \( n \) is the \textbf{degree} of the polynomial and \( a_n \) is the \textbf{leading coefficient}.
    \item The \textbf{degree} is the highest exponent of \( x \).
    \item The \textbf{leading coefficient} is the coefficient of the term with the highest degree.
    \item If a polynomial has an \textbf{even degree} and a \textbf{negative leading coefficient}, its \textbf{end behavior} is:
    \begin{itemize}
        \item As \( x \to +\infty \), \( P(x) \to -\infty \) (\(\downarrow\)).
        \item As \( x \to -\infty \), \( P(x) \to -\infty \) (\(\downarrow\)).
    \end{itemize}
    \item This means that both ends of the graph go downward.
    \item To sketch the graph, we:
    \begin{itemize}
        \item Find the \textbf{intercepts} (\textbf{x-intercepts} and \textbf{y-intercept}).
        \item Identify the \textbf{multiplicity} of each zero (even multiplicity means the graph \textbf{touches} the x-axis, odd multiplicity means it \textbf{crosses} the x-axis).
        \item Use the \textbf{Table of Signs} to determine the function’s behavior between intercepts.
        \item Consider the \textbf{end behavior} to complete the sketch.
    \end{itemize}
\end{itemize}
