\begin{center}
\textbf{Lesson 4.09: Calculating Deciles for Grouped Data}
\end{center}

%\vspace*{1ex}
\vspace*{-1.5ex}

\noindent \textbf{Deciles} divide a data set into ten equal parts, each containing \(10\%\) of the data. For grouped data, deciles are calculated using the formula:  
\[
D_k = L + \dfrac{\left(\dfrac{kN}{10} - CF\right)}{f} \cdot h
\]
where:  
\begin{itemize}
    \item \(D_k\) = \(k\)-th decile (\(k = 1, 2, \dots, 9\))
    \item \(L\) = Lower boundary of the decile class
    \item \(N\) = Total frequency
    \item \(CF\) = Cumulative frequency before the decile class
    \item \(f\) = Frequency of the decile class
    \item \(h\) = Class width
\end{itemize}

% \noindent\textbf{Procedure:}  
% \begin{enumerate}
%     \item Determine the decile class by locating the cumulative frequency interval that contains \(\dfrac{kN}{10}\).  
%     \item Identify the lower boundary (\(L\)), cumulative frequency before the decile class (\(CF\)), frequency of the decile class (\(f\)), and class width (\(h\)).  
%     \item Substitute the values into the formula to calculate \(D_k\).  
% \end{enumerate}
