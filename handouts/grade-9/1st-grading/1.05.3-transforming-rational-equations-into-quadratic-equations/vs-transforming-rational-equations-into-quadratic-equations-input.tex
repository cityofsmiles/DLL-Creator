\begin{center}
\textbf{Lesson 1.5.3: Transforming Rational Equations into Quadratic Equations}
\end{center}

%\vspace*{1ex}
\textbf{Rational algebraic equations:} equations that involve fractions containing algebraic expressions in the numerator and/or denominator; are typically in the form \( \dfrac{P(x)}{Q(x)} = 0 \), where \( P(x) \) and \( Q(x) \) are polynomials.

\textbf{How to Transform Rational Equations into Quadratic Equations}

%\begin{multicols}{2}
\begin{enumerate}[label = \color{blue}\arabic*. ]
%1ex
\item \textbf{Identify Rational Equations:} Recognize equations that involve rational expressions. Look for equations where variables are present in the denominator or numerator of fractions.
%2
\item \textbf{Find a Common Denominator:} Simplify the equation by finding a common denominator for all fractions. Identify the least common multiple (LCM) of the denominators and rewrite each fraction with the common denominator.
%3
\item \textbf{Eliminate Fractions:} Remove fractions from the equation to make it easier to work with. Multiply both sides of the equation by the common denominator to eliminate fractions.
%4
\item \textbf{Simplify the Equation:} Simplify the resulting equation to reveal its quadratic nature. Combine like terms and rearrange the equation to set it equal to zero.
%5
\item \textbf{Identify Quadratic Form:} Recognize the quadratic form of the equation. Ensure the equation resembles $ ax^2 + bx + c = 0 $, where $ a $, $ b $, and $ c $ are constants and $ a \neq 0 $.
\end{enumerate}
%\end{multicols} 



