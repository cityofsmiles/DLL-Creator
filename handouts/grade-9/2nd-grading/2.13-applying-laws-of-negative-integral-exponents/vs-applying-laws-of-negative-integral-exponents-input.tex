\begin{center}
\textbf{Lesson 2.13: Applying Laws of Negative Integral Exponents}
\end{center}

\vspace*{1ex}
%\vspace*{-1.5ex}

%\begin{multicols}{2}
\setlist{nolistsep}
\begin{enumerate}[noitemsep, label = \color{blue}\arabic*. ]
    \item Negative Exponent Rule: \( a^{-m} = \dfrac{1}{a^m}, \quad \text{where } a \neq 0 \)
    \item Product of Powers: \( a^m \cdot a^n = a^{m+n} \)
    \item Quotient of Powers:  \( \dfrac{a^m}{a^n} = a^{m-n}, \quad \text{where } a \neq 0 \)
    \item Power of a Power: \( (a^m)^n = a^{m \cdot n} \)
    \item Power of a Product: \( (ab)^m = a^m \cdot b^m \)
    \item Power of a Quotient: \( \left( \dfrac{a}{b} \right)^m = \dfrac{a^m}{b^m}, \quad \text{where } b \neq 0 \)
\end{enumerate}
% \end{multicols}


\noindent\textbf{How to Apply the Laws of Negative Exponents}
%\begin{multicols}{2}
\setlist{nolistsep}
\begin{enumerate}[noitemsep, label = \color{blue}\arabic*. ]
    \item Recognize and apply the negative exponent rule.
    \item Use other exponent rules (product, quotient, power) to simplify expressions.
    \item Convert negative exponents to positive by rewriting them as reciprocals.
\end{enumerate}


