\begin{center}
\textbf{Lesson 1.08.2: Dividing Polynomials Using Synthetic Division}
\end{center}

%\vspace*{1ex}
\vspace*{-1.5ex}

\begin{itemize}
    \item \textbf{Synthetic division} is a shortcut method for dividing a polynomial by a linear divisor of the form \(x - c\).
    \item It is an efficient alternative to \textbf{long division} when the divisor is a first-degree polynomial.
    \item The steps for synthetic division are:
          \begin{itemize}
              \item Write the coefficients of the dividend.
              \item Identify the root \(c\) from the divisor \(x - c\).
              \item Perform synthetic division by following the multiplication and addition steps.
              \item The final row contains the coefficients of the quotient and the remainder.
          \end{itemize}
    \item If the remainder is zero, then \(x - c\) is a factor of the dividend.
\end{itemize}