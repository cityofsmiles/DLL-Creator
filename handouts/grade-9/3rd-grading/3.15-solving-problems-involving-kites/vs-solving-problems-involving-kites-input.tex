\begin{center}
\textbf{Lesson 3.15: Solving Problems Involving Kites}
\end{center}

%\vspace*{1ex}
\vspace*{-1.5ex}


\noindent\textbf{Step-by-Step Procedure for Solving Kite Problems}

\begin{enumerate}[label=\color{blue}\arabic*.]
    \item \textbf{Identify Given Information:} Read the problem carefully and list the values provided, such as lengths of sides or diagonals and the relationships between angles or sides.
    \item \textbf{Apply Kite Properties:} 
    \begin{itemize}
        \item Use the property that the diagonals are perpendicular.
        \item Recall that one diagonal bisects the other.
        \item If you need to find the area of the kite, use the formula \(A = \frac{1}{2} d_1 d_2\).
    \end{itemize}
    \item \textbf{Set Up an Equation:} 
    \begin{itemize}
        \item For area-related problems, plug in the diagonal lengths into the area formula and solve for the missing variable.
        \item For other properties, use congruent sides or angles, and set up equations based on the given relationships.
    \end{itemize}
    \item \textbf{Solve for Unknowns:} Use algebraic techniques to solve for the unknown values.
    \item \textbf{Verify Your Solution:} Substitute your answer back into the equations to check that it satisfies all given conditions.
\end{enumerate}
