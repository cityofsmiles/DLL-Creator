\begin{center}
\textbf{Lesson 4.23: Estimating Number of Samples Using Cochran's Formula}
\end{center}

%\vspace*{1ex}
\vspace*{-1.5ex}

\noindent\textbf{Cochran’s Formula:} used to estimate the minimum sample size required for a study or survey to ensure reliable results. It accounts for the population proportion (\(p\)) and the margin of error (\(e\)) while assuming a confidence level (\(Z\)).

{\centering $ n_0 = \dfrac{Z^2 \cdot p \cdot q}{e^2} $\par}

\noindent where:
\begin{itemize}
    \item \(n_0\): Initial sample size
    \item \(Z\): Z-value (based on the confidence level, e.g., \(1.96\) for 95\%)
    \item \(p\): Proportion of the population with the characteristic of interest (as a decimal)
    \item \(q = 1 - p\): Proportion of the population without the characteristic
    \item \(e\): Margin of error (as a decimal)
\end{itemize}

% \noindent\textbf{Procedure:}
% \begin{enumerate}
%     \item Identify the confidence level and find the corresponding \(Z\)-value.
%     \item Determine the estimated proportion of the population (\(p\)) and calculate \(q = 1 - p\).
%     \item Specify the margin of error (\(e\)).
%     \item Substitute the values into Cochran’s Formula.
%     \item Solve for \(n_0\), and round to the nearest whole number if necessary.
% \end{enumerate}
