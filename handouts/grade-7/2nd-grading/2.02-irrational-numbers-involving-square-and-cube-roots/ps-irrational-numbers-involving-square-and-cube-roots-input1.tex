%\vspace{1ex}
\vspace{0.3ex}
\noindent\textbf{Practice Exercises 2.02}

%\vspace{0.75ex}
\vspace{0.2ex}

A. Write \emph{R} if the given number is rational or \emph{I} if it is irrational.
\begin{multicols}{2}
\setlist{nolistsep}
\begin{enumerate}[noitemsep, label = \color{blue}\arabic*. ]
\item \(\dfrac{\sqrt{5}}{\sqrt{5}}\)
\item \(0.888\ldots\)
\item \(\sqrt{36}\)
\item \(2\pi\)
\item \(\sqrt{2}\)
\item \(\dfrac{7}{3}\)
\item \(4.25\)
\item \(\sqrt{49}\)
\item \(\sqrt[{\scriptstyle 3}]{-8}\)
\item \(3.14159\ldots\)  
\end{enumerate}
\end{multicols}

B. Find two whole numbers between which each irrational number lies.
\begin{multicols}{2}
\setlist{nolistsep}
\begin{enumerate}[noitemsep, label = \color{blue}\arabic*. ]
\item \(\sqrt{24}\)
\item \(\sqrt[3]{15}\)
\item \(\sqrt{30}\)
\item \(\sqrt[3]{20}\)
\item \(\sqrt{50}\)
\item \(\sqrt[3]{10}\)
\item \(\sqrt{18}\)
\item \(\sqrt[3]{9}\)
\item \(\sqrt{8}\)
\item \(\sqrt[3]{5}\)
\end{enumerate}
\end{multicols}
