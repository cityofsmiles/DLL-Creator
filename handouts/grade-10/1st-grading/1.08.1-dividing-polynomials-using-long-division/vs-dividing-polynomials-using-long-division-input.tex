2\begin{center}
\textbf{Lesson 1.08.1: Dividing Polynomials Using Long Division}
\end{center}

%\vspace*{1ex}
\vspace*{-1.5ex}

\begin{itemize}
    \item \textbf{Polynomial long division} is a method used to divide a polynomial by another polynomial of lower or equal degree.
    \item The process is similar to numerical long division and follows these steps:
          \begin{itemize}
              \item Divide the first term of the dividend by the first term of the divisor.
              \item Multiply the entire divisor by the result from the previous step.
              \item Subtract the obtained product from the dividend.
              \item Bring down the next term and repeat the process until no terms remain.
          \end{itemize}
    \item The result of the division may include a \textbf{quotient} and a \textbf{remainder}.
    \item If the remainder is zero, the divisor is a factor of the dividend.
\end{itemize}
