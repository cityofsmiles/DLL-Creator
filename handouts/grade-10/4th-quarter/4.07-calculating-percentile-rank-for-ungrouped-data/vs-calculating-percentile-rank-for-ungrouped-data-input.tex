\begin{center}
\textbf{Lesson 4.07: Calculating Percentile Rank for Ungrouped Data}
\end{center}

%\vspace*{1ex}
\vspace*{-1.5ex}

\noindent\textbf{Percentile Rank (PR)} of a value in a data set indicates the percentage of data values below that value

\noindent\textbf{Formula:}  
\[
PR = \dfrac{F}{N} \times 100
\]
where:  
\begin{itemize}
    \item \(PR\) = Percentile rank
    \item \(F\) = Number of data values less than the given value
    \item \(N\) = Total number of data values in the set 
\end{itemize}

\noindent\textbf{Procedure:}  
\begin{enumerate}
    \item Arrange the data set in ascending order.  
    \item Identify the number of data values (\(F\)) that are less than the given value.  
    \item Use the formula to calculate \(PR\).  
    \item Express the result as a percentage.  
\end{enumerate}
