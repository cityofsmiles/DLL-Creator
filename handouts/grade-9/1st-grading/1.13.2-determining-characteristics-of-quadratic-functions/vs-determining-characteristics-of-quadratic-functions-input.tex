\begin{center}
\textbf{Lesson 1.13.2: Determining Characteristics of Quadratic Functions}
\end{center}

%\vspace*{1ex}

\begin{enumerate}[label=\color{blue}\arabic*. ]
    \item \textbf{Vertex}: the minimum or the maximum point on the quadratic function. If the quadratic function is expressed in the form of $ y = a(x-h)^{2} + k $, the vertex is the point $(h,k)$. If it is in the form of $ y = ax^{2} + bx + c $, use the formula $  h = -\dfrac{b}{2a}$ and $k = f(h)$. 
    
    \item \textbf{Axis of Symmetry}: divides the graph into two parts such that one half of the graph is a reflection of the other half. The equation of this line is $x = h$. 
    
    \item \textbf{Direction of Opening}: If $a > 0$, the parabola opens upwards and has a minimum point. If $a < 0$, it opens downwards and has a maximum point. 
    
    \item \textbf{Range}: depends on whether the parabola opens upward or downward. If it opens upward, then the range is the set $\{y: y \geq k\}$; if it opens downward then the range is the set $\{y: y \leq k\}$. 
    
    \item \textbf{Intercepts}: The $x$-intercepts (zeros) can be found by solving the equation $f(x) = 0$. The $y$-intercept is given by the value of $f(0)$.
    
    \item \textbf{Domain}: The domain of a quadratic function is the set of real numbers. 
\end{enumerate}

 



