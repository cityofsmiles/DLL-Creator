\begin{center}
\textbf{Lesson 1.12.2: Describing and Finding the Rational Roots of Polynomial Equations of Degree Greater than 2}
\end{center}

%\vspace*{1ex}
\vspace*{-1.5ex}

\begin{itemize}
    \item A \textbf{polynomial equation} is an equation involving a polynomial expression set to zero.
    \item The \textbf{rational root theorem} states that any rational root, expressed as $\dfrac{p}{q}$, must have $p$ as a factor of the constant term and $q$ as a factor of the leading coefficient.
    \item To find the \textbf{rational zeros}:
    \begin{itemize}
        \item List the possible rational roots using the Rational Root Theorem.
        \item Use synthetic division or polynomial division to test possible roots.
        \item If a root is found, factorize the polynomial further or use synthetic division iteratively.
    \end{itemize}
    \item A polynomial equation may have \textbf{multiple rational roots, irrational roots, or complex roots}.
\end{itemize}
