\begin{center}
\textbf{Lesson 2.12: Finding Arc and Inscribed Angle Measures in a Circle}
\end{center}

%\vspace*{1ex}
\vspace*{-1.5ex}


\begin{itemize}
    \item An \textbf{inscribed angle} is an angle whose \textbf{vertex} is on the circle and whose sides contain \textbf{chords} of the circle.
    \item The \textbf{intercepted arc} of an inscribed angle is the portion of the circle that lies inside the angle.
    \item The measure of an \textbf{inscribed angle} is always \textbf{half} the measure of its \textbf{intercepted arc}:

{\centering $
    \text{Inscribed Angle} = \dfrac{1}{2} \times \text{Intercepted Arc}
$\par}
    \item If two inscribed angles intercept the same arc, they are \textbf{congruent} (they have the same measure).
    \item An \textbf{angle inscribed in a semicircle} is always a \textbf{right angle} ($90^\circ$).
    \item The measure of a \textbf{minor arc} is less than $180^\circ$, while the measure of a \textbf{major arc} is greater than $180^\circ$.
    \item The sum of all arcs in a circle is $360^\circ$.
\end{itemize}

