\vspace{1ex}
\noindent\textbf{Activity 1.2}

\vspace{0.75ex}

For each data set below, calculate the mean, median, and mode. Then, draw a conclusion about what the data tells you.

\begin{enumerate} 
\item Data Set: 5, 10, 15, 20, 25, 30 \\ 
\textit{Hint:} How do the mean and median relate? What does this say about symmetry?

\item Data Set: 6, 6, 8, 10, 12, 12, 12 \\
 \textit{Hint:} What does the mode tell you about frequency?

\item Data Set: 50, 60, 70, 80, 300 \\
 \textit{Hint:} Does one value skew the data? Which measure is most affected?

\item Data Set: 20, 40, 60, 80, 100, 120, 140 \\ 
\textit{Hint:} Is the data evenly spaced? What does that imply?

\item Data Set: 3, 3, 5, 5, 7, 9 \\
 \textit{Hint:} Are there two modes? What does this say about the data?

\item Data Set: 90, 100, 110, 120, 130 \\
 \textit{Hint:} How do the measures of central tendency compare?

\item Data Set: 14, 18, 18, 21, 24, 27 \\ 
\textit{Hint:} What value occurs most often?

\item Data Set: 1, 2, 4, 6, 9, 11, 13, 16 \\ 
\textit{Hint:} If there is no mode, what does that suggest?

%\item Data Set: 25, 35, 45, 55, 120 \\
 %\textit{Hint:} What impact does the highest number have on the average?

%\item Data Set: 5, 10, 10, 15, 20, 20, 25 \\
% \textit{Hint:} How do repeated values shape the interpretation of the data? 
 \end{enumerate}


