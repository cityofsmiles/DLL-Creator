\begin{center}
\textbf{Lesson 2.18: Theorem on Angles Formed by the Intersection of a Tangent and a Secant Outside a Circle}
\end{center}

%\vspace*{1ex}
\vspace*{-1.5ex}

\begin{itemize}
    \item A \textbf{tangent} is a line that touches a circle at exactly one point, called the \textbf{point of tangency}.
    \item A \textbf{secant} is a line that intersects a circle at two points.
    \item When a tangent and a secant intersect outside a circle, they form an \textbf{exterior angle}.
    \item The measure of the exterior angle formed by the tangent and the secant is given by the formula:

{\centering $
    \text{Exterior Angle} = \dfrac{1}{2} (\text{major intercepted arc} - \text{minor intercepted arc}).
$\par}
    \item This theorem allows us to solve problems involving unknown angles or arc measures in circle geometry.
\end{itemize}
