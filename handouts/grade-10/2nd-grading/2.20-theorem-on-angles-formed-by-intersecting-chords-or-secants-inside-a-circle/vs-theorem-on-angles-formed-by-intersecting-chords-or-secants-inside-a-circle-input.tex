\begin{center}
\textbf{Lesson 2.20: Theorem on Angles Formed by Intersecting Chords or Secants Inside a Circle}
\end{center}

%\vspace*{1ex}
\vspace*{-1.5ex}

\begin{itemize}
    \item A \textbf{chord} is a segment whose endpoints lie on the circle.
    \item A \textbf{secant} is a line that intersects a circle at two points.
    \item When two chords or secants intersect inside a circle, they form an \textbf{interior angle}.
    \item The measure of the angle formed by two intersecting chords or secants inside a circle is given by the formula:

{\centering $
    \text{Interior Angle} = \dfrac{1}{2} (\text{sum of the measures of the intercepted arcs}).
$\par}
    \item This theorem helps find unknown angle measures in problems involving intersecting chords or secants.
\end{itemize}
