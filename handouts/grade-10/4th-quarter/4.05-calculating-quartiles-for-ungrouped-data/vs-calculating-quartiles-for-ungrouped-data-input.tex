\begin{center}
\textbf{Lesson 4.05: Calculating Quartiles for Ungrouped Data}
\end{center}

%\vspace*{1ex}
\vspace*{-1.5ex}

\noindent \textbf{Quartiles} divide a data set into four equal parts. These are:

\begin{itemize}
        \item \(Q_1\) (First Quartile): Represents the 25\% mark of the data.
        \item \(Q_2\) (Second Quartile): Represents the 50\% mark (median) of the data.
        \item \(Q_3\) (Third Quartile): Represents the 75\% mark of the data.
    \end{itemize}
  
\noindent\textbf{Procedure:}  
\begin{enumerate}
    \item Arrange the data in ascending order.  
    \item Calculate the positions of the quartiles using the formulas:
    \begin{itemize}
        \item \(Q_1\) position: \(\dfrac{n+1}{4}\)  
        \item \(Q_2\) position: \(\dfrac{n+1}{2}\)  
        \item \(Q_3\) position: \(\dfrac{3(n+1)}{4}\)
    \end{itemize}
    \item Identify the values corresponding to the quartile positions.  
    \item If the position is a whole number, the value is directly taken. If it is a decimal, interpolate between the nearest values.  
\end{enumerate}
