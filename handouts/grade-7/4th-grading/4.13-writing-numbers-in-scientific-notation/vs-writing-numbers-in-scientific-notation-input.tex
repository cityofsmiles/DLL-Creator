\begin{center}
\textbf{Lesson 4.13: Writing Numbers in Scientific Notation}
\end{center}

%\vspace*{1ex}
\vspace*{-1.5ex}

\noindent \textbf{ Scientific Notation}: a way to express very large or very small numbers using powers of 10

 \noindent A number in scientific notation is written as \( a \times 10^n \), where:  
    \begin{enumerate}
        \item \( 1 \leq |a| < 10 \) 
        \item \( n \) is an integer 
    \end{enumerate}

\noindent To convert from standard form to scientific notation:
    \begin{enumerate}
        \item Move the decimal point to make the coefficient between 1 and 10.  
        \item Count the number of places the decimal was moved; this determines the exponent.  
        \item If the decimal is moved to the left, the exponent is positive. If moved to the right, the exponent is negative.  
    \end{enumerate}

\noindent To convert from scientific notation to standard form, multiply the coefficient by \( 10^n \) (move the decimal point \( n \) places to the right for positive \( n \) or to the left for negative \( n \)).  
 

