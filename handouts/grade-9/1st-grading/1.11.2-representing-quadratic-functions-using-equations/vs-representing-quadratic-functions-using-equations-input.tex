\begin{center}
\textbf{Lesson 1.11.2: Representing Quadratic Functions Using Equations}
\end{center}

\textbf{Forms of Quadratic Equations}

A quadratic function can be represented in various forms, each providing different insights:
\begin{itemize}
%1
\item \textbf{Standard Form}

The standard form of a quadratic equation is:

\vspace*{-1.7ex}
\[ f(x) = ax^2 + bx + c \]
\vspace*{-3ex}

where \(a\), \(b\), and \(c\) are constants, and \(a \neq 0\). This form is useful for identifying the coefficient values directly and applying the quadratic formula for solving equations.
%2
\item \textbf{Vertex Form}

The vertex form of a quadratic equation is:

\vspace*{-1.7ex}
\[ f(x) = a(x-h)^2 + k \]
\vspace*{-3ex}

where \((h, k)\) is the vertex of the parabola. This form is beneficial for easily finding the vertex and sketching the graph.
%3
\item \textbf{Factored Form}

The factored form of a quadratic equation is:

\vspace*{-1.7ex}
\[ f(x) = a(x - r_1)(x - r_2) \]
\vspace*{-3ex}

where \(r_1\) and \(r_2\) are the roots of the quadratic equation. This form is helpful for solving the equation by factoring and for understanding the x-intercepts of the graph.
\end{itemize}








