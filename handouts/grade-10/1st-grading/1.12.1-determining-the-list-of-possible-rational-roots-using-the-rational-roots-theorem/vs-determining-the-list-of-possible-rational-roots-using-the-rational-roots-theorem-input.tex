\begin{center}
\textbf{Lesson 1.12.1: Determining the List of Possible Rational Roots Using the Rational Roots Theorem}
\end{center}

%\vspace*{1ex}
\vspace*{-1.5ex}

\begin{itemize}
    \item A \textbf{polynomial equation} is an expression set equal to zero containing variables raised to whole number exponents.
    \item The \textbf{Rational Roots Theorem} provides a way to determine all possible rational zeros of a polynomial equation.
    \item The possible rational roots are given by \( \pm \dfrac{p}{q} \), where:
    \begin{itemize}
        \item \( p \) represents the factors of the \textbf{constant term}.
        \item \( q \) represents the factors of the \textbf{leading coefficient}.
    \end{itemize}
    \item Not all values in the list are actual roots; they must be tested using substitution or division.
\end{itemize}