\begin{center}
\textbf{Lesson 1.10.1: Finding the Factors of Polynomial Expressions of Degree 2}
\end{center}

%\vspace*{1ex}
\vspace*{-1.5ex}

\begin{itemize}
    \item A \textbf{polynomial} is an algebraic expression consisting of variables and coefficients.
    \item \textbf{Factoring} is the process of expressing a polynomial as a product of its factors.
    \item A polynomial of degree at most 2 can be factored using various methods:
    \begin{itemize}
        \item \textbf{Greatest Common Factor (GCF)} – factoring out the highest common factor.
        \item \textbf{Factoring by Grouping} – grouping terms to factor common binomial factors.
        \item \textbf{Trinomials} – expressing a quadratic expression as a product of two binomials.
        \item \textbf{Difference of Squares} – factoring expressions of the form $a^2 - b^2$ as $(a - b)(a + b)$.
    \end{itemize}
    \item Factoring is useful in solving \textbf{polynomial equations} by setting factors equal to zero.
\end{itemize}