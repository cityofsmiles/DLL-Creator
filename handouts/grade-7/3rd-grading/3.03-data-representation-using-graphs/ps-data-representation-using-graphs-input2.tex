%\vspace{1ex}
\vspace{0.3ex}
\noindent\textbf{Activity 3.03}

%\vspace{0.75ex}
\vspace{0.2ex}

\begin{enumerate}[label=\color{blue}\arabic*. , noitemsep]
    \item \textbf{Histogram} \\
    The number of hours spent on homework per week by a group of students is recorded as follows:  
    \(
    2, 3, 4, 4, 5, 5, 6, 6, 7, 8, 8, 9
    \)  
    Construct a histogram to represent this data.

    \item \textbf{Pie Chart} \\
    A family spends their monthly budget as follows:
    \begin{itemize}
        \item Food: 30\%
        \item Housing: 25\%
        \item Transportation: 20\%
        \item Savings: 25\%
    \end{itemize}
    Draw a pie chart to represent this budget distribution.

    \item \textbf{Bar Graph} \\
    A sports club records the number of members in each of its activities:
    \begin{itemize}
        \item Basketball: 20
        \item Soccer: 15
        \item Tennis: 10
        \item Swimming: 5
    \end{itemize}
    Create a bar graph for the data above.

    \item \textbf{Stem-and-Leaf Plot} \\
    Here are the test scores of a group of students in a math exam: 55, 58, 62, 64, 69, 71, 72, 75, 78, 80.  
    Represent this data in a stem-and-leaf plot.

    % \item \textbf{Pie Chart Interpretation} \\
    % Based on the pie chart provided, showing the distribution of various food items in a pantry (e.g., 40\% grains, 30\% vegetables, 20\% fruits, 10\% others), answer the following:
    % \begin{enumerate}[label=(\alph*)]
    %     \item Which category occupies the most space?
    %     \item What percentage of the pantry is fruits and others combined?
    % \end{enumerate}
\end{enumerate}
