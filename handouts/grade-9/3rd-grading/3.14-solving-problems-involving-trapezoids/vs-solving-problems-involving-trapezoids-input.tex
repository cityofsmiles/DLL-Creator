\begin{center}
\textbf{Lesson 3.14: Solving Problems Involving Trapezoids}
\end{center}

%\vspace*{1ex}
\vspace*{-1.5ex}

\noindent\textbf{Step-by-Step Procedure}

\begin{enumerate}[label=\color{blue}\arabic*.]
    \item \textbf{Identify the Known Values:} Read the problem carefully to identify the given values for the trapezoid’s bases, midsegment, height, or other sides.
    
    \item \textbf{Apply the Midsegment Formula:} For a trapezoid with bases \(a\) and \(b\), the length of the midsegment \(m\) is:
      
    {\centering $ 
    m = \dfrac{a + b}{2}
    $\par}
  
    Use this formula if you need to find the midsegment or one of the base lengths.
    
  
    \item \textbf{Set Up Equations for Unknowns:} Substitute the given values into the formulas and solve for the unknowns.
    
    \item \textbf{Check and Verify Your Solution:} Ensure that your answer is reasonable by substituting back into the original equations. Make sure all units are consistent.
\end{enumerate}
