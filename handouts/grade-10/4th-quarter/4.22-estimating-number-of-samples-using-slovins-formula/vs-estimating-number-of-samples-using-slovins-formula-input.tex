\begin{center}
\textbf{Lesson 4.22: Estimating Number of Samples Using Slovin's Formula}
\end{center}

%\vspace*{1ex}
\vspace*{-1.5ex}

\noindent\textbf{Slovin’s Formula:} a statistical tool used to estimate the sample size required for a given population size and margin of error%. It is particularly useful when conducting surveys or research where surveying the entire population is impractical or too costly. By specifying the acceptable margin of error (\(e\)), Slovin’s Formula ensures that the sample size is representative of the population, enabling accurate and reliable conclusions.

{\centering $ n = \dfrac{N}{1 + N e^2} $\par}

\noindent where:
\begin{itemize}
    \item \(n\): Sample size
    \item \(N\): Population size
    \item \(e\): Margin of error (as a decimal)
\end{itemize}

% \noindent\textbf{Procedure:}
% \begin{enumerate}
%     \item Identify the total population size (\(N\)).
%     \item Determine the desired margin of error (\(e\)).
%     \item Substitute the values of \(N\) and \(e\) into Slovin’s Formula.
%     \item Solve for \(n\), the required sample size.
%     \item Round \(n\) to the nearest whole number if necessary.
% \end{enumerate}
