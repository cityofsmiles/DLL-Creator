\begin{center}
\textbf{Lesson 3.31: Applying Triangle Angle Bisector Theorem to Solve Problems Involving Proportions}
\end{center}

%\vspace*{1ex}
\vspace*{-1.5ex}

\noindent\textbf{Triangle Angle Bisector Theorem}
\begin{itemize}%[label=\color{blue}\arabic*.]
    \item The angle bisector of a triangle divides the opposite side into two segments that are proportional to the other two sides of the triangle.
    \item If a bisector divides side \(BC\) of \(\triangle ABC\) into segments \(BD\) and \(DC\), then:
      
    {\centering $ 
    \dfrac{BD}{DC} = \dfrac{AB}{AC}.
     $\par}
\end{itemize}

\noindent\textbf{Steps for Solving Problems Using the Triangle Angle Bisector Theorem}
\begin{enumerate}
    \item \textbf{Identify the Given Segments:} Note the side lengths and the segments into which the opposite side is divided.
    \item \textbf{Set Up a Proportion:} Use the Triangle Angle Bisector Theorem to write a proportion.
    \item \textbf{Solve for the Unknown:} Use cross-multiplication or algebraic techniques to find the missing length.
    \item \textbf{Check Your Work:} Verify that your solution satisfies the proportion.
\end{enumerate}
