\begin{center}
\textbf{Lesson 1.03.3: Finding the Sum of Terms in an Arithmetic Sequence}
\end{center}

%\vspace*{1ex}
\vspace*{-1.5ex}

\begin{itemize}
    \item An \textbf{arithmetic sequence} is a sequence in which the difference between consecutive terms is constant.
    \item The sum of the first $n$ terms of an arithmetic sequence, called the \textbf{arithmetic series}, is given by:

{\centering $ S_n = \dfrac{n}{2} (2a + (n-1)d)  $\par}
      %\[ S_n = \dfrac{n}{2} (2a + (n-1)d) \]
    or equivalently,

{\centering $  S_n = \dfrac{n}{2} (a + l)  $\par}
    %\[ S_n = \dfrac{n}{2} (a + l) \]
\noindent     where:
    \begin{itemize}
        \item $S_n$ is the sum of the first $n$ terms,
        \item $a$ is the first term,
        \item $l$ is the last term,
        \item $d$ is the common difference, and
        \item $n$ is the number of terms.
    \end{itemize}
    \item The formula allows for quick computation of the sum without listing all terms.
\end{itemize}
