 \begin{center}
\textbf{Lesson 3.4: Graphing Solutions of Linear Inequalities on a Number Line}
\end{center}

\vspace*{1ex}

The solutions to linear inequalities are ranges of numbers, which we can represent visually on a number line.

\noindent\textbf{Steps for Graphing Linear Inequalities:}
%\begin{multicols}{2}
\setlist{nolistsep}
\begin{enumerate}[noitemsep, label = \color{blue}\arabic*. ]
    \item \textbf{Solve the inequality:} Follow the usual steps for solving the inequality to isolate the variable on one side.
    \item \textbf{Draw the number line:} Draw a horizontal number line and mark a few key numbers, including the solution.
    \item \textbf{Plot the boundary point:} If the inequality includes $<$ or $>$, plot an open circle at the boundary point. If it includes $\leq$ or $\geq$, plot a closed circle.
    \item \textbf{Shade the region:} Shade the part of the number line that represents the solution to the inequality. If the inequality is $<$ or $\leq$, shade to the left. If it is $>$ or $\geq$, shade to the right.
\end{enumerate}


