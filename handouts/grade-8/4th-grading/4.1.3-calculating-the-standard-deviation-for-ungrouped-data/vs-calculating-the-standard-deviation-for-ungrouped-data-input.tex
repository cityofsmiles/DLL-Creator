 \begin{center}
\textbf{Lesson 4.1: Measures of Variability for Ungrouped Data}
\end{center}

\vspace*{1ex}

\noindent\textbf{Standard Deviation}: The standard deviation is the square root of the variance and is given by:

{\centering $ s = \sqrt{\dfrac{\sum (x_i - \bar{x})^2}{n}} $\par}

where:
\begin{itemize}
    \item $x_i$ are the data values,
    \item $\bar{x}$ is the mean of the data,
    \item $n$ is the number of data values, and
    \item $(x_i - \bar{x})^2$ represents the squared deviations of each data value from the mean.
\end{itemize}

\noindent\textbf{Steps for Calculating Standard Deviation}

\begin{enumerate}
    \item \textbf{Find the mean:} Add all the data values and divide by the number of values.
    \item \textbf{Find the deviations:} Subtract the mean from each data value.
    \item \textbf{Square the deviations:} Square each deviation.
    \item \textbf{Find the mean of the squared deviations:} Add all the squared deviations and divide by the number of data values (this is called the variance).
    \item \textbf{Take the square root:} Find the square root of the variance to get the standard deviation.
\end{enumerate}


