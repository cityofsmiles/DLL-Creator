\vspace{0.3ex}
\noindent  \textbf{Practice Exercises 3.15.1}

\vspace{0.2ex}

A. Identify whether each ordered pair is a solution to the given inequality. Write \emph{YES} if it is or \emph{NO} if it is not.

%\vspace{-2ex}

\begin{multicols}{3}
\begin{enumerate}%[label = \arabic*. ]
%1
\item $x+y > -1$
\begin{enumerate}%[label = \alph*. ]
%1
\item (-1, 2)
%2
\item (0, 0)
%3
\item (-3, 2) 
\end{enumerate}   

\vfill
\columnbreak

%2
\item $2x-y \geq 3$
\begin{enumerate}%[label = \alph*. ]
%1
\item (2, 1)
%2
\item (2, 0)
%3
\item $\big(\displaystyle \frac{1}{2}, 2\big)$
\end{enumerate}  

\vfill
\columnbreak

%3
\item $3x+2y \leq 5$
\begin{enumerate}%[label = \alph*. ]
%1
\item (4, 2)
%2
\item $\big(-\displaystyle \frac{1}{2}, -3\big)$
%3
\item (-5, 2)
\end{enumerate}  
\end{enumerate}  
\end{multicols} 

%\vspace{-1ex}

B. Translate the following situations into mathematical phrases.
\begin{enumerate}%[label = \arabic*. ]
%1
\item The sum of two numbers is less than 7.
%2
\item The difference of two numbers is greater than 2.
%3
\item Thrice a number is less than or equal to another number.
%4
\item Nicole bought 2 earrings ($e$) and 3 bracelets ($b$) and paid not more than \pesos 1,000.00.
%5 
\item Twice the number of mango ($m$) exceeds thrice the number of guava ($g$).
\end{enumerate}  

