\begin{center}
\textbf{Lesson 2.15: Illustrating Secants, Tangents, Segments, and Sectors of a Circle}
\end{center}

%\vspace*{1ex}
\vspace*{-1.5ex}

\begin{itemize}
    \item A \textbf{secant} is a line that intersects a circle at two points.
    \item A \textbf{tangent} is a line that touches a circle at exactly one point.
    \item A \textbf{segment of a circle} is the region bounded by a chord and the corresponding arc.
    \item A \textbf{sector of a circle} is the region enclosed by two radii and the intercepted arc.
    \item The formula for the \textbf{area of a sector} is 

{\centering $
    A = \pi r^2 \times \dfrac{\theta}{360^\circ}
$\par}
    where $r$ is the radius and $\theta$ is the central angle in degrees.
    \item The formula for the \textbf{area of a segment} is 

{\centering $
    A = \pi r^2 \times \dfrac{\theta}{360^\circ} - \dfrac{1}{2} r^2 \sin\theta
$\par}
    where $r$ is the radius and $\theta$ is the central angle in degrees.
\end{itemize}
