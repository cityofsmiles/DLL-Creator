\begin{center}
\textbf{Lesson 2.17: Theorem on Angles Formed by Intersecting Secants Outside a Circle}
\end{center}

%\vspace*{1ex}
\vspace*{-1.5ex}

\begin{itemize}
    \item A \textbf{secant} is a line that intersects a circle at two points.
    \item When two secants intersect outside a circle, they form an \textbf{exterior angle}.
    \item The measure of the exterior angle formed by two secants is given by the formula:

{\centering $
    \text{Exterior Angle} = \dfrac{1}{2} (\text{major intercepted arc} - \text{minor intercepted arc}).
$\par}
    \item The exterior angle is always equal to half the difference of the intercepted arcs.
    \item This theorem can be used to find unknown angles or arc measures in geometric problems involving circles.
\end{itemize}
