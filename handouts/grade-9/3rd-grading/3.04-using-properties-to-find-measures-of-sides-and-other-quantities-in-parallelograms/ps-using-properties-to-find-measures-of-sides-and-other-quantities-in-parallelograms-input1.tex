%\vspace{1ex}
\vspace{0.3ex}
\noindent\textbf{Practice Exercises 3.04}

%\vspace{0.75ex}
\vspace{0.2ex}

\noindent Use the properties of parallelograms to find the measure of the indicated side or quantity.

\begin{enumerate}[label=\color{blue}\arabic*.]
    \item In parallelogram \(ABCD\), \(AB = (2x + 5)\) and \(CD = (3x - 7)\). Find \(AB\).
%    \item In parallelogram \(PQRS\), \(PQ = (4y - 10)\) and \(SR = (3y + 2)\). Determine the length of \(PQ\).
 %   \item In parallelogram \(EFGH\), \(EF = (5z + 3)\) and \(GH = (2z + 15)\). Calculate the length of \(EF\).
  %  \item In parallelogram \(WXYZ\), \(WX = (6a + 10)\) and \(YZ = (4a + 20)\). Find the length of \(WX\).
    \item In parallelogram \(JKLM\), \(JK = (7b - 5)\) and \(ML = (3b + 15)\). Determine the length of \(JK\).
    \item In parallelogram \(MNOP\), diagonal \(MO\) and \(NP\) intersect at \(Q\). If \(MQ = (3c + 12)\) and \(OQ = (5c - 8)\), calculate \(MO\).
  %  \item In parallelogram \(QRST\), \(QR = (4d + 10)\) and \(ST = (2d + 30)\). Find the length of \(QR\).
    \item In parallelogram \(UVWX\), \(UV = (6e - 10)\) and \(WX = (4e + 20)\). Determine the length of \(UV\).
    \item In parallelogram \(ABCD\), diagonal \(AC\) and \(BD\) intersect at \(E\). If \(AE = (2f + 15)\) and \(CE = (4f - 5)\), calculate \(AC\).
   % \item In parallelogram \(PQRS\), \(PS = (3g + 25)\) and \(RQ = (5g - 15)\). Find the length of \(PS\).
\end{enumerate}
