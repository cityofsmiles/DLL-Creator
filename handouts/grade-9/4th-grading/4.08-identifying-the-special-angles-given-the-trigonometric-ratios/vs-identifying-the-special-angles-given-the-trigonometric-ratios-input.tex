\begin{center}
\textbf{Lesson 4.08: Identifying the Special Angles Given the Trigonometric Ratios}
\end{center}

%\vspace*{1ex}
\vspace*{-1.5ex}

\noindent \textbf{Special Angles:} The special angles in trigonometry are \(0^\circ\), \(30^\circ\), \(45^\circ\), \(60^\circ\), and \(90^\circ\). These angles have specific trigonometric ratio values that are often used to solve problems.

\noindent  \textbf{Key Ratios:}
    \begin{align*}
        \sin 30^\circ &= \dfrac{1}{2} & \cos 30^\circ &= \dfrac{\sqrt{3}}{2} & \tan 30^\circ &= \dfrac{\sqrt{3}}{3} \\
        \sin 45^\circ &= \dfrac{\sqrt{2}}{2} & \cos 45^\circ &= \dfrac{\sqrt{2}}{2} & \tan 45^\circ &= 1 \\
        \sin 60^\circ &= \dfrac{\sqrt{3}}{2} & \cos 60^\circ &= \dfrac{1}{2} & \tan 60^\circ &= \sqrt{3}
    \end{align*}

% \noindent \textbf{Procedure:}
%     \begin{enumerate}
%         \item Identify the given trigonometric ratio.
%         \item Compare the given ratio to the standard values for special angles.
%         \item Determine the angle that corresponds to the given ratio.
%     \end{enumerate}
