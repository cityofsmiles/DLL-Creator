\vspace{1ex}
\noindent\textbf{Practice Exercises 1.2}

\vspace{0.75ex}

For each data set below, calculate the mean, median, and mode. Then, draw a conclusion about what the data tells you.

\begin{enumerate}
    \item Data Set: 10, 12, 14, 16, 18, 20 \\
    \textit{Hint:} Is the mean different from the median? What does this suggest about the data distribution?

    \item Data Set: 3, 3, 5, 7, 9, 9, 9 \\
    \textit{Hint:} How does the mode help you understand the most common value?

    \item Data Set: 100, 200, 300, 400, 1000 \\
    \textit{Hint:} How does the mean compare to the median, and what does this tell you about the presence of outliers?

    \item Data Set: 25, 50, 75, 100, 125, 150, 175 \\
    \textit{Hint:} Consider the distribution of the data. Is the median a better measure of central tendency?

    \item Data Set: 4, 4, 6, 6, 8, 10 \\
    \textit{Hint:} Does the data have multiple modes? What does that indicate about the distribution?

    \item Data Set: 45, 55, 65, 75, 85 \\
    \textit{Hint:} Compare the mean and median. What conclusion can you draw?

    \item Data Set: 8, 12, 12, 15, 20, 22 \\
    \textit{Hint:} How does the mode help you understand the most frequent value?

  %  \item Data Set: 2, 3, 5, 7, 11, 13, 17, 19 \\
  %  \textit{Hint:} Is there a mode? If not, what does this tell you?

  %  \item Data Set: 30, 40, 50, 60, 100 \\
  %  \textit{Hint:} How does the presence of an outlier affect the mean?

  %  \item Data Set: 10, 20, 20, 30, 40, 40, 50 \\
 %   \textit{Hint:} How does the mode help in understanding the distribution of the data?
\end{enumerate}
