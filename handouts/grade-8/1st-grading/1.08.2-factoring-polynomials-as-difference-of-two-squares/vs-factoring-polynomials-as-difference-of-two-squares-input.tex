\begin{center}
\textbf{Lesson 1.8.2: Factoring Polynomials as Difference of Two Squares}
\end{center}

\vspace*{1ex}

\textbf{Perfect Square:} When a polynomial is multiplied by itself, then it is a perfect square.

\textbf{Difference of Two Squares:} a squared polynomial subtracted from another squared polynomial 

\textbf{Formula:}
The factored form of a polynomial that is a difference of two squares is the sum and difference of the square roots of the first and last terms. 
 
In symbols,  
\vspace*{-1ex} 
$$\displaystyle a^{2}-b^{2} = (a+b)(a-b) $$
\vspace*{-6ex} 
\begin{center}
%\vspace*{-1ex} 
or
\end{center}
\vspace*{-1ex} 
$$ \text{1st}^2 - \text{2nd}^2 = (\text{1st} + \text{2nd})(\text{1st} - \text{2nd}) $$	
