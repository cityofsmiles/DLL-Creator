%\vspace{1ex}
\vspace{0.3ex}
\noindent\textbf{Activity 3.03}

%\vspace{0.75ex}
\vspace{0.2ex}

\noindent Use the properties of parallelograms to solve for the measure of the indicated angle.

\begin{enumerate}[label=\color{blue}\arabic*.]
    \item In parallelogram \(ABCD\), \(\angle BAD = (2a+5)^\circ\) and \(\angle BCD = (3a-15)^\circ\). Calculate \(\angle BAD\).
    \item In parallelogram \(PQRS\), \(\angle QPS = (4x + 10)^\circ\) and \(\angle SRQ = (5x - 20)^\circ\). Determine \(\angle QPS\).
    \item In parallelogram \(EFGH\), \(\angle EFG = (3b + 30)^\circ\) and \(\angle FGH = (2b + 50)^\circ\). Calculate \(\angle EFG\).
    \item In parallelogram \(WXYZ\), \(\angle WXY = (6c - 14)^\circ\) and \(\angle XYZ = (3c + 5)^\circ\). Find \(\angle WXY\).
    \item In parallelogram \(JKLM\), \(\angle JKL = (4d + 25)^\circ\) and \(\angle LMJ = (5d - 5)^\circ\). Determine \(\angle JKL\).
    \item In parallelogram \(MNOP\), \(\angle NOP = (7e - 20)^\circ\) and \(\angle OPM = (5e + 44)^\circ\). Calculate \(\angle NOP\).
    \item In parallelogram \(RSTU\), \(\angle RST = (5f + 15)^\circ\) and \(\angle TUR = (3f + 45)^\circ\). Find \(\angle RST\).
    \item In parallelogram \(VWXZ\), \(\angle VWX = (2g + 62)^\circ\) and \(\angle WXZ = (4g - 20)^\circ\). Determine \(\angle VWX\).
    \item In parallelogram \(ABCD\), \(\angle DAB = (3h + 10)^\circ\) and \(\angle BCD = (5h - 40)^\circ\). Calculate \(\angle DAB\).
    \item In parallelogram \(PQRS\), \(\angle RSP = (6j - 30)^\circ\) and \(\angle SPQ = (4j + 10)^\circ\). Find \(\angle RSP\).
\end{enumerate}
