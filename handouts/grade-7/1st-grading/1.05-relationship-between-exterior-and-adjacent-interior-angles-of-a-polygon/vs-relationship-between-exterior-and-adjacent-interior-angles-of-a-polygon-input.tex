\begin{center}
\textbf{Lesson 1.05: Relationship Between Exterior and Adjacent Interior Angles of a Polygon}
\end{center}

%\vspace*{1ex}
\vspace*{-1.5ex}

\begin{itemize}
    \item An exterior angle of a polygon is formed when one side of the polygon is extended.
    \item The exterior angle and the adjacent interior angle of a polygon are supplementary. This means that the sum of these two angles equals 180°.
    \item The relationship between the exterior and interior angles can be used to find the unknown angles in a polygon.
    \item For a regular polygon, all exterior angles are equal, and the sum of all exterior angles of any polygon is always 360°.
    \item The exterior angle can be used to determine the number of sides in a regular polygon using the formula:
      
{\centering $
    \text{Exterior angle} = \dfrac{360^\circ}{n}
    $\par}
  
    where \(n\) is the number of sides of the polygon.
\end{itemize}
