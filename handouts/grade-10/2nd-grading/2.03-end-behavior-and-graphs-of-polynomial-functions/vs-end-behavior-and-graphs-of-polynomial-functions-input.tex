\begin{center}
\textbf{Lesson 2.03: End Behavior and Graphs of Polynomial Functions}
\end{center}

%\vspace*{1ex}
\vspace*{-1.5ex}

\begin{itemize}
    \item The \textbf{Leading Coefficient Test} helps determine the \textbf{end behavior} of a polynomial function based on its \textbf{degree} and \textbf{leading coefficient}.
    \item The \textbf{degree} of a polynomial function is the highest exponent of the variable.
    \item The \textbf{leading coefficient} is the coefficient of the term with the highest degree.
    \item The end behavior of a polynomial function describes how the graph behaves as \( x \to \pm\infty \).
    \item The end behavior follows these rules:
    \begin{itemize}
        \item If the \textbf{degree} is \textbf{even}:
        \begin{itemize}
            \item If the \textbf{leading coefficient} is \textbf{positive}, both ends of the graph rise (\(\uparrow, \uparrow\)).
            \item If the \textbf{leading coefficient} is \textbf{negative}, both ends of the graph fall (\(\downarrow, \downarrow\)).
        \end{itemize}
        \item If the \textbf{degree} is \textbf{odd}:
        \begin{itemize}
            \item If the \textbf{leading coefficient} is \textbf{positive}, the left end falls and the right end rises (\(\downarrow, \uparrow\)).
            \item If the \textbf{leading coefficient} is \textbf{negative}, the left end rises and the right end falls (\(\uparrow, \downarrow\)).
        \end{itemize}
    \end{itemize}
    \item Using this test, we can sketch the \textbf{possible shape} of the graph.
\end{itemize}
