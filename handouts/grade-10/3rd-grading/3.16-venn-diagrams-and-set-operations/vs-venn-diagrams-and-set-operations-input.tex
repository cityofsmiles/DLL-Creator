\begin{center}
\textbf{Lesson 3.16: Venn Diagrams and Set Operations}
\end{center}

%\vspace*{1ex}
\vspace*{-1.5ex}

\noindent\textbf{Key Concepts}

\begin{enumerate}[label=\color{blue}\arabic*.]
    \item \textbf{Set} – A collection of distinct objects or elements.
    \item \textbf{Subset} – A set whose elements are all contained in another set.
    \item \textbf{Union} (\(A \cup B\)) – The set of elements that are in \(A\), or in \(B\), or in both.
    \item \textbf{Intersection} (\(A \cap B\)) – The set of elements that are in both \(A\) and \(B\).
    \item \textbf{Complement} (\(A^c\)) – The set of elements not in \(A\).
    \item \textbf{Difference} (\(A - B\)) – The set of elements in \(A\) but not in \(B\).
    \item \textbf{Compound Events} – Events that involve two or more outcomes, such as the union or intersection of two events.
\end{enumerate}

% \noindent\textbf{Step-by-Step Procedure}

% \begin{enumerate}
%     \item Identify the sets involved in the problem.
%     \item Determine the set operation required: union, intersection, difference, or complement.
%     \item Draw the Venn diagram, labeling the sets and their relationship.
%     \item Apply the set operation to the Venn diagram to solve the problem.
%     \item For compound events, use the Venn diagram to represent the outcomes and calculate the probability if necessary.
% \end{enumerate}
