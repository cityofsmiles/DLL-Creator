\begin{center}
\textbf{Lesson 2.26: Deriving the Midpoint Formula and Finding Midpoints of Line Segments}
\end{center}

%\vspace*{1ex}
\vspace*{-1.5ex}

\begin{itemize}
    \item The \textbf{midpoint} of a line segment is the point that divides the segment into two equal parts.
    \item The \textbf{midpoint formula} is derived from the concept of averaging the coordinates of two given endpoints.
    \item If the endpoints of a segment are $A(x_1, y_1)$ and $B(x_2, y_2)$, then the coordinates of the midpoint $M$ are given by:
    \[
    M \left( \dfrac{x_1 + x_2}{2}, \dfrac{y_1 + y_2}{2} \right)
    \]
    \item The midpoint formula is useful in coordinate geometry, physics, and real-world applications such as navigation and design.
\end{itemize}
