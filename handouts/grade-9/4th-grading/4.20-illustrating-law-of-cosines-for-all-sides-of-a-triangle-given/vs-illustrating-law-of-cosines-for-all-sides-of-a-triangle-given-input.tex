\begin{center}
\textbf{Lesson 4.20: Illustrating Law of Cosines for All Sides of a Triangle Given}
\end{center}

%\vspace*{1ex}
\vspace*{-1.5ex}

The \textbf{Law of Cosines} can be used to find the measure of an angle in a triangle when all three sides are known. The formula is:  

{\centering $
\cos C = \dfrac{a^2 + b^2 - c^2}{2ab}
$\par}

\noindent where:
\begin{itemize}
    \item \(a\), \(b\), and \(c\) are the sides of the triangle.
    \item \(C\) is the angle opposite side \(c\), which we aim to find.
\end{itemize}

% \noindent\textbf{Step-by-Step Procedure:}
% \begin{enumerate}
%     \item \textbf{Identify the known values.} Note the lengths of all three sides of the triangle (\(a\), \(b\), and \(c\)).
%     \item \textbf{Write the Law of Cosines formula.
%     \item \textbf{Substitute the known values.} Plug in the lengths of \(a\), \(b\), and \(c\).
%     \item \textbf{Calculate \(\cos C\).} Perform the operations in the numerator and denominator.
%     \item \textbf{Find \(C\).} Use a calculator to find the inverse cosine (\(\cos^{-1}\)) of the result.
%     \item \textbf{Check your answer.} Ensure the angle is reasonable for the given triangle.
% \end{enumerate}
