\begin{center}
\textbf{Lesson 2.10: Describing Sets, Subsets, Unions, and Intersections of Sets}
\end{center}

%\vspace*{1ex}
\vspace*{-1.5ex}

\noindent\textbf{Set:} a well-defined collection of objects

\noindent\textbf{Element \(\in\):} an object that belongs to a set

\noindent\textbf{Ways to Indicate a Set:}
%\begin{multicols}{2}
\setlist{nolistsep}
\begin{enumerate}[noitemsep, label = \color{blue}\arabic*. ]
\item Roster method: the elements of a set are listed inside braces, separated by commas
\item Rule method: a phrase that describes each element of a set is enclosed in braces
\end{enumerate}
% \end{multicols}


\noindent A set \(S\) is a \textbf{subset} of a set \(A\) if all elements of set \(S\) are also elements of set \(A\).

\noindent\textbf{Universal set} (\(U\)): contains all objects under consideration

\noindent\textbf{Empty or null set} (\(\emptyset\)): a set that contains no element

\noindent\textbf{Cardinality of set \(A\)} (\(|A|\)): the number of elements in A

The cardinality of a set is \textbf{finite} if its elements can be listed completely.
The cardinality is \textbf{infinite} and an ellipsis (\ldots) is used to indicate the infinite list of elements. 


