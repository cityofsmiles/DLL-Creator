%\vspace{1ex}
\vspace{0.3ex}
\noindent\textbf{Practice Exercises 3.01}

%\vspace{0.75ex}
\vspace{0.2ex}

Use the Fundamental Counting Principle to find the number of possible arrangements for each scenario.

\begin{enumerate}[label=\color{blue}\arabic*.]
    \item A password consists of 3 letters followed by 2 digits. How many different passwords can be created if there are 26 letters and 10 digits available?
    \item A pizza place offers 3 sizes, 4 crust types, and 5 topping options. How many different types of pizza can be made?
    \item A restaurant offers 5 types of drinks, 3 appetizers, and 4 entrees. How many different meal combinations can a customer choose?
    \item A combination lock uses 3 numbers, each between 0 and 9. How many different combinations are possible?
    \item A class has 4 different books, and a student needs to select 2 books to read. How many different ways can the student choose 2 books?
    % \item A concert hall has 6 different sections, 10 rows in each section, and 15 seats in each row. How many total seats are there in the hall?
    % \item A team has 5 forwards, 3 midfielders, and 4 defenders. How many ways can a coach select one forward, one midfielder, and one defender to start the game?
    % \item A student needs to choose a shirt (4 options), a pair of pants (3 options), and shoes (2 options). How many different outfits can the student wear?
    % \item A license plate consists of 2 letters followed by 3 digits. How many different license plates can be made if there are 26 letters and 10 digits available?
    % \item A group of 3 friends wants to sit in a row at a movie theater with 8 available seats. How many different seating arrangements are possible?
\end{enumerate}
