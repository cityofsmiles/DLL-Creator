\begin{center}
\textbf{Lesson 1.02.2: Finding the Missing Term Common Difference and Number of Terms in an Arithmetic Sequence}
\end{center}

%\vspace*{1ex}
\vspace*{-1.5ex}

\begin{itemize}
    \item An \textbf{arithmetic sequence} is a sequence of numbers in which the difference between consecutive terms is constant.
    \item The \textbf{common difference}, denoted as $d$, is found using the formula:

      {\centering $ d = a_2 - a_1  $\par}
      %\[ d = a_2 - a_1 \]
    \item The \textbf{nth term} of an arithmetic sequence is given by:

      {\centering $  a_n = a_1 + (n-1)d  $\par}
      %\[ a_n = a_1 + (n-1)d \]
    \item To find a \textbf{missing term}, use the general formula or determine the pattern based on given terms.
    \item The number of terms $n$ in an arithmetic sequence is found by rearranging the nth term formula:

{\centering $ n = \dfrac{a_n - a_1}{d} + 1  $\par}
      %\[ n = \dfrac{a_n - a_1}{d} + 1 \]
\end{itemize}
