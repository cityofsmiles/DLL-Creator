%\vspace{1ex}
\vspace{0.3ex}
\noindent\textbf{Practice Exercises 3.13}

%\vspace{0.75ex}
\vspace{0.2ex}

Solve the following problems.

\begin{enumerate}
    \item In a parallelogram \(ABCD\), \(AB = 12\) cm, and \(BC = 9\) cm. Find the length of \(CD\) and \(DA\).
    \item In parallelogram \(EFGH\), the measure of \(\angle E = 70^\circ\). Find the measures of \(\angle F\), \(\angle G\), and \(\angle H\).
    \item The diagonals of parallelogram \(JKLM\) intersect at point \(O\). If \(JO = 6\) cm, find the length of \(JL\).
    \item In parallelogram \(PQRS\), \(PQ = 8x + 2\) and \(RS = 5x + 14\). Find the value of \(x\) and the lengths of \(PQ\) and \(RS\).
    \item In parallelogram \(WXYZ\), \(WX = 3y - 4\) and \(YZ = y + 10\). Find the value of \(y\) and the lengths of \(WX\) and \(YZ\).
    \item In a parallelogram, the measure of one angle is \(110^\circ\). What are the measures of the other three angles?
    \item In parallelogram \(ABCD\), the diagonals \(AC\) and \(BD\) are equal, and \(AC = 16\) cm. What type of parallelogram is \(ABCD\)?
    \item In parallelogram \(KLMN\), \(KL = 7a - 1\) and \(MN = 4a + 8\). Find the value of \(a\) and the lengths of \(KL\) and \(MN\).
    \item In parallelogram \(QRST\), \(QR = 12\) cm, and \(RS\) is twice the length of \(QR\). Find the lengths of \(QR\), \(RS\), \(ST\), and \(TQ\).
    \item In parallelogram \(UVWX\), the diagonals \(UW\) and \(VX\) intersect at point \(O\). If \(UO = 8\) cm and \(VX = 30\) cm, find the length of \(UW\) and \(VO\).
\end{enumerate}
