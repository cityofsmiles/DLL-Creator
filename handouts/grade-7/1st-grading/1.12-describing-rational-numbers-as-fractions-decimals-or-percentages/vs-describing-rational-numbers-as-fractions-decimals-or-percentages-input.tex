\begin{center}
\textbf{Lesson 1.12: Describing Rational Numbers as Fractions Decimals or Percentages}
\end{center}

%\vspace*{1ex}
\vspace*{-1.5ex}

\begin{itemize}
    \item A \textbf{rational number} is any number that can be expressed as a fraction \(\dfrac{a}{b}\), where \(a\) and \(b\) are integers and \(b \neq 0\).
    \item Rational numbers can be written in three forms:
    \begin{itemize}
        \item \textbf{Fraction}: A ratio of two numbers (e.g., \(\dfrac{3}{4}\)).
        \item \textbf{Decimal}: A fraction written in decimal notation (e.g., \(0.75\)).
        \item \textbf{Percentage}: A fraction out of 100, written with a percent sign (e.g., \(75\%\)).
    \end{itemize}
    \item To convert between forms:
    \begin{itemize}
        \item \textbf{Fraction to Decimal}: Divide the numerator by the denominator.
        \item \textbf{Decimal to Fraction}: Write the decimal as a fraction and simplify.
        \item \textbf{Fraction to Percentage}: Multiply by 100 and add the percent symbol.
        \item \textbf{Percentage to Fraction}: Write over 100 and simplify.
        \item \textbf{Percentage to Decimal}: Divide by 100.
        \item \textbf{Decimal to Percentage}: Multiply by 100.
    \end{itemize}
\end{itemize}
