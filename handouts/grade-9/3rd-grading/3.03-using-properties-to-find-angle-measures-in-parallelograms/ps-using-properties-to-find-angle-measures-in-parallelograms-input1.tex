%\vspace{1ex}
\vspace{0.3ex}
\noindent\textbf{Practice Exercises 3.03}

%\vspace{0.75ex}
\vspace{0.2ex}

\noindent Use the properties of parallelograms to solve for the measure of the indicated angle.

\begin{enumerate}[label=\color{blue}\arabic*.]
    \item In parallelogram \(ABCD\), \(\angle DAB = (3p + 20)^\circ\) and \(\angle BCD = (2p + 40)^\circ\). Calculate \(\angle DAB\).
    \item In parallelogram \(PQRS\), \(\angle QPS = (5q - 15)^\circ\) and \(\angle SRQ = (3q + 5)^\circ\). Determine \(\angle QPS\).
    \item In parallelogram \(EFGH\), \(\angle EFG = (4r + 10)^\circ\) and \(\angle GHE = (5r - 30)^\circ\). Calculate \(\angle EFG\).
    \item In parallelogram \(WXYZ\), \(\angle WXY = (7s - 10)^\circ\) and \(\angle XYZ = (2s + 37)^\circ\). Find \(\angle WXY\).
    \item In parallelogram \(JKLM\), \(\angle JKL = (6t + 11)^\circ\) and \(\angle KLM = (3t + 25)^\circ\). Determine \(\angle JKL\).
    %\item In parallelogram \(MNOP\), \(\angle NOP = (8u - 15)^\circ\) and \(\angle PON = (5u + 10)^\circ\). Calculate \(\angle NOP\).
   % \item In parallelogram \(RSTU\), \(\angle RST = (9v + 20)^\circ\) and \(\angle TUR = (4v - 25)^\circ\). Find \(\angle RST\).
  %  \item In parallelogram \(VWXZ\), \(\angle VWX = (3w + 45)^\circ\) and \(\angle XZV = (6w - 15)^\circ\). Determine \(\angle VWX\).
 %   \item In parallelogram \(ABCD\), \(\angle DAB = (2x + 55)^\circ\) and \(\angle BCD = (7x - 35)^\circ\). Calculate \(\angle DAB\).
%    \item In parallelogram \(PQRS\), \(\angle QRP = (4y + 40)^\circ\) and \(\angle SPQ = (5y - 10)^\circ\). Find \(\angle QRP\).
\end{enumerate}
