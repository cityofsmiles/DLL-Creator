%\vspace{1ex}
\vspace{0.3ex}
\noindent\textbf{Practice Exercises 3.30}

%\vspace{0.75ex}
\vspace{0.2ex}

Use the Converse Triangle Proportionality Theorem to solve the following problems.

\begin{enumerate}[label=\color{blue}\arabic*.]
    \item In \(\triangle ABC\), \(D\) and \(E\) divide \(AB\) and \(AC\), respectively. If \(AD = 6\), \(DB = 4\), \(AE = 9\), and \(EC = 6\), is \(DE \parallel BC\)?
    \item In \(\triangle XYZ\), points \(P\) and \(Q\) lie on \(XY\) and \(XZ\), respectively. If \(XP = 5\), \(PY = 10\), \(XQ = 6\), and \(QZ = 12\), is \(PQ \parallel YZ\)?
    \item In \(\triangle DEF\), points \(M\) and \(N\) divide \(DE\) and \(DF\). If \(DM = 8\), \(ME = 12\), \(DN = 10\), and \(NF = 14\), is \(MN \parallel EF\)?
    \item In \(\triangle PQR\), points \(S\) and \(T\) lie on \(PQ\) and \(PR\), respectively. If \(PS = 4\), \(SQ = 6\), \(PT = 5\), and \(TR = 10\), is \(ST \parallel QR\)?
    \item In \(\triangle LMN\), points \(X\) and \(Y\) are on \(LM\) and \(LN\), respectively. If \(LX = 9\), \(XM = 3\), \(LY = 12\), and \(YN = 4\), is \(XY \parallel MN\)?
    % \item In \(\triangle ABC\), \(D\) and \(E\) divide \(AB\) and \(AC\). If \(AD = 7\), \(DB = 5\), \(AE = 10\), and \(EC = 7\), is \(DE \parallel BC\)?
    % \item In \(\triangle PQR\), points \(S\) and \(T\) lie on \(PQ\) and \(PR\). If \(PS = 6\), \(SQ = 9\), \(PT = 8\), and \(TR = 12\), is \(ST \parallel QR\)?
    % \item In \(\triangle XYZ\), \(M\) and \(N\) divide \(XY\) and \(XZ\). If \(XM = 5\), \(MY = 10\), \(XN = 7.5\), and \(NZ = 15\), is \(MN \parallel YZ\)?
    % \item In \(\triangle DEF\), points \(P\) and \(Q\) are on \(DE\) and \(DF\), respectively. If \(DP = 4\), \(PE = 8\), \(DQ = 6\), and \(QF = 12\), is \(PQ \parallel EF\)?
    % \item In \(\triangle ABC\), points \(M\) and \(N\) divide \(AB\) and \(AC\). If \(AM = 10\), \(MB = 5\), \(AN = 15\), and \(NC = 7.5\), is \(MN \parallel BC\)?
\end{enumerate}
