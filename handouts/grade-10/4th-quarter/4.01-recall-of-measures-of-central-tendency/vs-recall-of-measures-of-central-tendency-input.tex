\begin{center}
\textbf{Lesson 4.01: Recall of Measures of Central Tendency}
\end{center}

%\vspace*{1ex}
\vspace*{-1.5ex}

\noindent \textbf{Mean:} the sum of all values divided by the number of values

{\centering $
    \text{Mean} = \dfrac{\text{Sum of all data values}}{\text{Number of data values}}
$\par}
    
\noindent  \textbf{Median:} the middle value of an ordered data set
    \begin{itemize}
        \item If the number of values is odd, the median is the middle value.  
        \item If the number of values is even, the median is the average of the two middle values.
    \end{itemize}
    
\noindent  \textbf{Mode:} the value(s) that appear most frequently in the data set.  A data set may have one mode, more than one mode (multimodal), or no mode.


\noindent\textbf{Procedure:}  
\begin{enumerate}
    \item Arrange the data set in ascending order.  
    \item Compute the mean by adding all the data values and dividing by the total number of values.  
    \item Identify the median by locating the middle value(s).  
    \item Determine the mode by finding the most frequent value(s) in the data set.  
    \item Verify results and simplify calculations where necessary.
\end{enumerate}
