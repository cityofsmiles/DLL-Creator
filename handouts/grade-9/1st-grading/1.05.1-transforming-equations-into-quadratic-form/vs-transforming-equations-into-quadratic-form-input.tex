\begin{center}
\textbf{Lesson 1.5.1: Transforming Equations into Quadratic Form}
\end{center}

\vspace*{1ex}



%\vspace{-1ex}
%\begin{multicols}{2}
\begin{enumerate}[label = \color{blue}\arabic*. ]
%1
   \item \textbf{Recognize the Form:} The equation is quadratic in form if the exponent on the leading term is double the exponent on the middle term. 
%2
   \item \textbf{Choose the Right Substitution:} Simplify the equation using an appropriate substitution. If the equation involves terms like $x^4$, $x^2$, consider substituting $u = x^2$. This substitution should help simplify the powers of $x$ to create a quadratic in terms of $u$.
%3
   \item \textbf{Rewrite the Equation:} Transform the original equation using your substitution to create a new equation in quadratic form. Replace every instance of your substituted expression in the original equation. For example, if $u = x^2$, replace $x^4$ with $u^2$ and $x^2$ with $u$.
\end{enumerate}
%\end{multicols} 



