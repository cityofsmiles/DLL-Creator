\begin{center}
\textbf{Lesson 2.16: Geometric Relationships Involving Tangents and Secants of a Circle}
\end{center}

%\vspace*{1ex}
\vspace*{-1.5ex}

\begin{itemize}
    \item A \textbf{tangent} to a circle is a line that touches the circle at exactly one point, called the \textbf{point of tangency}.
    \item A \textbf{secant} is a line that intersects a circle at two points.
    \item A \textbf{common external tangent} is a line that is tangent to two circles and does not intersect the segment joining their centers.
    \item A \textbf{common internal tangent} is a line that is tangent to two circles and intersects the segment joining their centers.
    \item If two tangents are drawn from an external point to a circle, then the tangents are \textbf{congruent}.
    \item The angle formed by two intersecting tangents to a circle is given by
      
    {\centering $ 
    \dfrac{1}{2} ( \text{major arc} - \text{minor arc}).
    $\par} 
\item If a secant and a tangent intersect outside a circle, then the measure of the angle formed is given by
  
    {\centering $
    \dfrac{1}{2} ( \text{major arc} - \text{minor arc}).
    $\par}
\item The \textbf{power of a point theorem} states that if a secant segment and a tangent segment are drawn from an external point, then
  
    {\centering $
    ( \text{tangent segment} )^2 = ( \text{whole secant} ) \times ( \text{external part of secant}).
    $\par}
\end{itemize}
