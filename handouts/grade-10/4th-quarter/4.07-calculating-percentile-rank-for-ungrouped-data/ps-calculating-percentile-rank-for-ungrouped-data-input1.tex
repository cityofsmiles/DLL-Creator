%\vspace{1ex}
\vspace{0.3ex}
\noindent\textbf{Practice Exercises 4.07}

%\vspace{0.75ex}
\vspace{0.2ex}

For each data set below, calculate the percentile rank of the given value.  

\begin{enumerate}
    \item Percentile rank of \(19\) in the data set: \(5, 10, 15, 18, 20, 25, 30\).  
    \item Percentile rank of \(58\) in the data set: \(10, 20, 30, 40, 50, 60, 70, 80, 90\).  
    \item Percentile rank of \(37\) in the data set: \(8, 16, 24, 32, 40, 48, 56, 64\).  
    \item Percentile rank of \(28\) in the data set: \(6, 12, 18, 24, 30, 36, 42, 48\).  
    \item Percentile rank of \(79\) in the data set: \(25, 50, 75, 100, 125, 150, 175\).  
    % \item Percentile rank of \(14\) in the data set: \(7, 10, 14, 18, 21, 25, 30\).  
    % \item Percentile rank of \(60\) in the data set: \(20, 40, 60, 80, 100, 120, 140\).  
    % \item Percentile rank of \(33\) in the data set: \(11, 22, 33, 44, 55, 66, 77, 88\).  
    % \item Percentile rank of \(48\) in the data set: \(12, 24, 36, 48, 60, 72, 84\).  
    % \item Percentile rank of \(29\) in the data set: \(5, 10, 15, 20, 25, 29, 35, 40\).  
\end{enumerate}
