 \begin{center}
\textbf{Lesson 3.3: Solving Linear Inequalities in One Variable}
\end{center}

\vspace*{1ex}

\noindent\textbf{Linear Inequality:} looks similar to a linear equation but uses inequality symbols like $<$, $>$, $\leq$, or $\geq$

\noindent\textbf{Steps to Solving Linear Inequalities:}
%\begin{multicols}{2}
\setlist{nolistsep}
\begin{enumerate}[noitemsep, label = \color{blue}\arabic*. ]
    \item \textbf{Simplify both sides:} Combine like terms and remove parentheses, just as with linear equations.
    \item \textbf{Move variable terms to one side:} Use addition or subtraction to get all variable terms on one side of the inequality.
    \item \textbf{Move constant terms to the other side:} Use addition or subtraction to get all constant terms on the other side of the inequality.
    \item \textbf{Solve for the variable:} Divide or multiply both sides by the coefficient of the variable, but \textbf{remember to reverse the inequality sign if you multiply or divide by a negative number}.
%    \item \textbf{Graph the solution on a number line:} Mark the boundary points and show the direction of the solution.
\end{enumerate}

