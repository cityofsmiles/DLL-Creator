\begin{center}
\textbf{Lesson 3.05: Indistinguishable and Circular Permutations}
\end{center}

%\vspace*{1ex}
\vspace*{-1.5ex}

\noindent \textbf{Indistinguishable Permutations:} If there are \(n\) objects where some are repeated, the formula to find the number of unique permutations is:

{\centering $ 
    P = \dfrac{n!}{n_1! \cdot n_2! \cdot \ldots \cdot n_k!},
 $\par}
    where \(n_1, n_2, \ldots, n_k\) are the frequencies of indistinguishable objects.

\noindent \textbf{Circular Permutations:} In a circular arrangement of \(n\) objects:
    \begin{itemize}
        \item If there is no fixed point (no distinction between starting points), the number of arrangements is \( P = (n-1)!\)
        
        \item If a starting point is fixed, the number of arrangements is \(P = n!\)
        
    \end{itemize}


% \noindent\textbf{Step-by-Step Procedure}
% \textbf{For Indistinguishable Permutations:}
% \begin{enumerate}
%     \item Count the total number of objects, \(n\).
%     \item Identify the frequencies of indistinguishable objects (\(n_1, n_2, \ldots, n_k\)).
%     \item Use the formula \(P = \frac{n!}{n_1! \cdot n_2! \cdot \cdots \cdot n_k!}\) to calculate the number of permutations.
% \end{enumerate}

% \textbf{For Circular Permutations:}
% \begin{enumerate}
%     \item Count the total number of objects, \(n\).
%     \item Decide if the arrangement has a fixed starting point or not.
%     \item Apply the formula:
%     \begin{itemize}
%         \item If no fixed point: \(P = (n-1)!\)
%         \item If fixed point: \(P = n!\)
%     \end{itemize}
% \end{enumerate}
