\begin{center}
\textbf{Lesson 3.13: Solving Problems Involving Parallelograms}
\end{center}

%\vspace*{1ex}
\vspace*{-1.5ex}

\textbf{Step 1: Identify Known Properties of Parallelograms}
\begin{itemize}
    \item Opposite sides are equal and parallel.
    \item Opposite angles are equal.
    \item Consecutive angles are supplementary.
    \item Diagonals bisect each other.
\end{itemize}

\textbf{Step 2: Set Up Your Diagram and Label Known Values}
Draw and label the parallelogram with any known values for side lengths, angles, or diagonal segments.

\textbf{Step 3: Solve for Unknown Side Lengths}
\begin{itemize}
    \item Use the Property of Opposite Sides: Opposite sides are equal.
   
    \item Set Up Equations for Variables: If sides are given as algebraic expressions, set them equal to each other and solve.
\end{itemize}

\textbf{Step 4: Solve for Unknown Angles}
\begin{itemize}
    \item Use the Property of Opposite Angles: Opposite angles are equal.
    
    \item Use the Supplementary Property of Consecutive Angles

\end{itemize}

\textbf{Step 5: Solve for Diagonal Segments}
\begin{itemize}
    \item Use the Diagonal Bisection Property: Each diagonal divides the other into two equal segments.
\end{itemize}

\textbf{Step 6: Verify and Check}
\begin{itemize}
    \item Substitute Your Solution Back: Ensure opposite sides are equal, opposite angles are equal, and consecutive angles add to \(180^\circ\).
\end{itemize}
