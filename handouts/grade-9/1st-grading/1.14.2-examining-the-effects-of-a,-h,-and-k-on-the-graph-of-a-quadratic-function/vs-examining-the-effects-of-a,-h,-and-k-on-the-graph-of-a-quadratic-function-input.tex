\begin{center}
\textbf{Lesson 1.14.2: Examining the Effects of a, h, and k on the Graph of a Quadratic Function}
\end{center}

\vspace*{1ex}

The graph of a quadratic function is a parabola, which can be transformed in various ways by changing the values of $a$, $h$, and $k$ in its vertex form:
\[ f(x) = a(x - h)^2 + k \]

\begin{itemize}
    \item If $a > 0$, the parabola opens upwards. The larger the value of $|a|$, the steeper the curve.
    \item If $a < 0$, the parabola opens downwards. The smaller the value of $|a|$, the steeper the curve.
%\end{itemize}


%\begin{itemize}
    \item If $h > 0$, the graph shifts $h$ units to the right.
    \item If $h < 0$, the graph shifts $|h|$ units to the left.
%\end{itemize}


%\begin{itemize}
    \item If $k > 0$, the graph shifts $k$ units upwards.
    \item If $k < 0$, the graph shifts $|k|$ units downwards.
\end{itemize}

