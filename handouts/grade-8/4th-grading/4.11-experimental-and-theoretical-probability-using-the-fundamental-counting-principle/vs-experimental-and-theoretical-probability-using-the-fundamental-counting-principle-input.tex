 \begin{center}
\textbf{Lesson 4.11: Experimental and Theoretical Probability Using the Fundamental Counting Principle}
\end{center}

\vspace*{-1.5ex}


\textbf{Experimental Probability:} the probability of an outcome of an event based on an experiment. The more trials done in an experiment, the closer the experimental probability gets to the theoretical probability.

%\vspace*{-1ex}\[
{\centering $ P(E) = \displaystyle \frac{\text{number of events}}{\text{total number of trials}} = \dfrac{f}{\sum f} $\par}

\textbf{Theoretical Probability:} the probability that a certain outcome will occur as determined through reasoning or calculation.

%\vspace*{-1ex}\[
{\centering $ P(E) = \displaystyle \frac{\text{number of ways the event can happen}}{\text{total number of possible outcomes}} = \dfrac{n(E)}{n(S)} $\par}
