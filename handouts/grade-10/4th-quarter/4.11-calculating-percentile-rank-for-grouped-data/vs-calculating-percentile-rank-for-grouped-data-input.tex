\begin{center}
\textbf{Lesson 4.11: Calculating Percentile Rank for Grouped Data}
\end{center}

%\vspace*{1ex}
\vspace*{-1.5ex}

The \textbf{percentile rank} of a score indicates the percentage of scores in a distribution that fall below it. For grouped data, the formula is:  

{\centering $PR = \dfrac{\left(F_b + \dfrac{f_i}{h} (x - L)\right)}{N} \cdot 100 $\par}

where:  
\begin{itemize}
    \item \(PR\) = Percentile rank  
    \item \(F_b\) = Cumulative frequency before the class containing the given value  
    \item \(f_i\) = Frequency of the class containing the given value  
    \item \(x\) = Given value  
    \item \(L\) = Lower boundary of the class containing the given value  
    \item \(h\) = Class width
    \item \(N\) = Total frequency  
\end{itemize}

% \noindent\textbf{Procedure:}  
% \begin{enumerate}
%     \item Identify the class interval containing the given value and determine its lower boundary (\(L\)), frequency (\(f_i\)), and cumulative frequency before it (\(F_b\)).  
%     \item Substitute the given value and identified parameters into the formula.  
%     \item Calculate the percentile rank (\(PR\)) and express it as a percentage.  
% \end{enumerate}
