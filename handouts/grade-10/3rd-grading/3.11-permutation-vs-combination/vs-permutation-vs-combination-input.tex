\begin{center}
\textbf{Lesson 3.11: Permutation vs Combination}
\end{center}

%\vspace*{1ex}
\vspace*{-1.5ex}

\noindent \textbf{Permutation} considers the arrangement of objects, where the order matters.

\noindent \textbf{Combination} does not consider the arrangement, where the order does not matter.

% \noindent Formula for permutations:   \( P(n, r) = \dfrac{n!}{(n-r)!}\)

% \noindent  Formula for combinations:   \( C(n, r) = \dfrac{n!}{r!(n-r)!}\)

\noindent  Relationship between permutations and combinations:  \(P(n, r) = C(n, r) \cdot r!\)

\noindent\textbf{Step-by-Step Procedure}

\begin{enumerate}
    \item Identify the total number of objects (\(n\)) and the number of objects to select (\(r\)).
    \item Decide if the problem is a permutation or combination based on whether the order matters.
    \item Apply the appropriate formula:

{\centering $ 
    P(n, r) = \dfrac{n!}{(n-r)!} \quad \text{or} \quad C(n, r) = \dfrac{n!}{r!(n-r)!}
    $\par}
  
    \item If converting between permutations and combinations, use the relationship:

{\centering $ 
    P(n, r) = C(n, r) \cdot r!
 $\par}      
\end{enumerate}
