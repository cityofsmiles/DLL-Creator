\begin{center}
\textbf{Lesson 4.18: Solving and Interpreting Real-Life Word Problems Involving Percentiles}
\end{center}

%\vspace*{1ex}
\vspace*{-1.5ex}

\noindent\textbf{Procedure for Solving Real-Life Word Problems Involving Percentiles:}  
\begin{enumerate}
    \item Arrange the data in ascending order.  
    \item Determine the position of the required percentile (\(P_k\)) using the formula.  
    % \[
    % P = \dfrac{k}{100} \cdot (n + 1)
    % \]
    % where \(P\) is the position, \(k\) is the percentile rank, and \(n\) is the total number of data points.  
    \item Locate the position in the dataset to find the corresponding percentile value.  
    \item Use the percentiles to answer the problem or interpret the data in the given context.  
    \item Clearly state the interpretation of the percentiles in relation to the scenario.
\end{enumerate}
