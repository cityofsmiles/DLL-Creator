\begin{center}
\textbf{Lesson 1.8.4: Factoring Quadratic Trinomials}
\end{center}

\vspace*{1ex}

\textbf{Quadratic Trinomial:} 
\begin{itemize}
%1
   \item When you multiply two binomial factors which are not identical, the result is a quadratic trinomial or general trinomial.
%2
   \item An expression in the form $ ax^2 + bx + c $ where $ a $, $ b $, and $ c $ are integers 
\end{itemize}
 
 
\textbf{Types of Quadratic Trinomials:}
		
		\begin{enumerate}
			\item Trinomial in the form of $ ax^2 + bx + c $, where $ a=1 $
			\item Trinomial in the form of $ ax^2 + bx + c $, where $ a \neq 1 $
		\end{enumerate}
		
\textbf{How to Factor Quadratic Trinomials:}
    	\begin{enumerate}
    		\item Factor out the greatest common monomial of all terms of the given expression.
    		\item Multiply the first term and the last term of the trinomial.  		
    		\item Get the possible factors of the product of the first term and the last term in such a way that their sum is equal to the second term of the trinomial.    		
    		\item Replace the middle term by the two factors.
    		\item Apply factoring by grouping.
    	\end{enumerate} 		
