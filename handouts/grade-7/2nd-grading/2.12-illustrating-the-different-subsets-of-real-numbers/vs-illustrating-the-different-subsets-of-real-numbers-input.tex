\begin{center}
\textbf{Lesson 2.12: Illustrating the Different Subsets of Real Numbers}
\end{center}

%\vspace*{1ex}
\vspace*{-1.5ex}

\noindent\textbf{Real Number System:} made up of the set of real numbers and its subsets

\noindent\textbf{Natural Numbers or Counting Numbers} (\(N\)): numbers used in counting objects

\(N = {1, 2, 3, 4, 5, \ldots}\)

\noindent\textbf{Whole Numbers} (\(W\)): natural numbers including 0

\(W = {0, 1, 2, 3, 4, 5, \ldots}\) 

\noindent\textbf{Integers} (\(Z\)): signed whole numbers

\(Z= {\ldots,-3, -2, -1, 0, 1, 2, 3,\ldots}\) 

\noindent\textbf{Rational Numbers} (\(Q\)): numbers that can be expressed as ratio of two integers and includes terminating and repeating decimals

\(Q = \{x| x \text{ can be expressed as a ratio of two integers}\}\) 

\noindent\textbf{Irrational Numbers} (\(Q’\)): numbers that cannot be expressed as ratio of two integers and include nonterminating and nonrepeating decimals

\(Q' = \{x| x \text{ cannot be expressed as a ratio of two integers}\}\)



