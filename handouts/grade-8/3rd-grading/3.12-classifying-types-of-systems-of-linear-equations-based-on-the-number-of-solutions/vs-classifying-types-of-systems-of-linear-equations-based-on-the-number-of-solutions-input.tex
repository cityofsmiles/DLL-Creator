 \begin{center}
\textbf{Lesson 3.12: Classifying Types of Systems of Linear Equations Based on the Number of Solutions}
\end{center}

\vspace*{-2ex}

A system of linear equations can have exactly one solution, infinitely many solutions, or no solution at all.

\noindent\textbf{Types of Solutions}:
    \begin{itemize}
        \item \textbf{One Solution}: The lines intersect at a single point.
        \item \textbf{No Solution}: The lines are parallel and never intersect.
        \item \textbf{Infinitely Many Solutions}: The lines are coincident (the same line) and have all points in common.
    \end{itemize}

\noindent\textbf{Steps for Classifying Systems}

\begin{enumerate}
    \item Rewrite both equations in slope-intercept form ($y = mx + b$), if necessary.
    \item Compare the slopes and intercepts of the two lines:
    \begin{itemize}
        \item If the slopes are different, the system has \textbf{one solution}.
        \item If the slopes are the same but the y-intercepts are different, the system has \textbf{no solution}.
        \item If the slopes and y-intercepts are the same, the system has \textbf{infinitely many solutions}.
    \end{itemize}
\end{enumerate}

