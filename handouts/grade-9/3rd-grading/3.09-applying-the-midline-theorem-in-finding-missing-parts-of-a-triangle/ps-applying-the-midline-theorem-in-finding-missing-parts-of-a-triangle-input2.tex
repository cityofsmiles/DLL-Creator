%\vspace{1ex}
\vspace{0.3ex}
\noindent\textbf{Activity 3.09}

%\vspace{0.75ex}
\vspace{0.2ex}

Use the Midline Theorem to find the missing lengths in each of the following triangles.
\begin{enumerate}[label=\color{blue}\arabic*.]
    \item In \(\bigtriangleup XYZ\), \(A\) and \(B\) are midpoints of \(XY\) and \(XZ\), respectively. If \(YZ = 24\), find \(AB\).
    \item In \(\bigtriangleup PQR\), \(C\) and \(D\) are midpoints of \(PQ\) and \(PR\). If \(CD = 14\), find \(QR\).
    \item In \(\bigtriangleup ABC\), \(F\) and \(G\) are midpoints of \(AB\) and \(AC\). If \(BC = 10\), find \(FG\).
    \item In \(\bigtriangleup DEF\), \(H\) and \(I\) are midpoints of \(DE\) and \(DF\). If \(EF = 18\), find \(HI\).
    \item In \(\bigtriangleup MNO\), \(J\) and \(K\) are midpoints of \(MN\) and \(MO\). If \(JK = 4\), find \(NO\).
    \item In \(\bigtriangleup STU\), \(L\) and \(M\) are midpoints of \(ST\) and \(SU\). If \(TU = 8\), find \(LM\).
    \item In \(\bigtriangleup VWX\), \(N\) and \(O\) are midpoints of \(VW\) and \(VX\). If \(NO = 6\), find \(WX\).
    \item In \(\bigtriangleup ABC\), \(P\) and \(Q\) are midpoints of \(AB\) and \(AC\). If \(BC = 22\), find \(PQ\).
    \item In \(\bigtriangleup JKL\), \(R\) and \(S\) are midpoints of \(JK\) and \(JL\). If \(RS = 9\), find \(KL\).
    \item In \(\bigtriangleup NOP\), \(T\) and \(U\) are midpoints of \(NO\) and \(NP\). If \(OP = 3\), find \(TU\).
\end{enumerate}
